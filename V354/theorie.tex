% This work is licensed under the Creative Commons
% Attribution-NonCommercial 3.0 Unported License. To view a copy of this
% license, visit http://creativecommons.org/licenses/by-nc/3.0/.

\section{Theorie}
\subsection{Gedämpfte Schwingungen}
Eine elektrische Schaltung, die einen \name{ohm}schen Widerstand R, eine
Induktivität L und eine Kapazität C enthält, welche eine Masche bilden,
wird als RLC-Kreis bezeichnet.  In Abb. \ref{fig:rlc-kreis} ist der
schematische Aufbau eines RLC-Kreises zu sehen.
%
\begin{figure}[h!!]
\centering
\includegraphics[height=3cm]{rlc_kreis}
\caption{Schaltbild eines RLC-Kreises}
\label{fig:rlc-kreis}
\end{figure}
%

Ein RLC-Kreis kann durch abwechselnde Speicherung von Energie im
magnetischen Feld der Induktivität und im elektrischen Feld der
Kapazität periodische Schwingungen der Stromstärke bzw. der Spannung
durchführen.

Um diese Schwingungen mathematisch zu beschreiben, werden die
\name{Kirchhoff}schen Regeln verwendet, welche sich aus den
\name{Maxwell}gleichungen ergeben.  Da es sich bei den Bauelemente des
RLC-Kreises um konzentrierte Bauelemente handelt, sind diese Regeln hier
anwendbar.  Die erste \name{Kirchhoff}sche Regel, die Knotenregel,
besagt, dass die Summe aller gerichteten Ströme in einem Knoten gleich
Null ist.  Die zweite \name{Kirchhoff}sche Regel, welche Maschenregel
genannt wird, legt dar, dass die Summe aller Spannungen einer Masche
verschwindet.

Diese Regeln werden auf die elektrische Schaltung in Abb.
\ref{fig:rlc-kreis} angewendet. Die Differentialgleichung für die
zeitabhängige Stromstärke I(t) ergibt sich zu
\begin{equation*}
\tdd{I}{t} + \frac{R}{L} \td{I}{t} + \frac{I}{LC} = 0.
\end{equation*}

Ein Exponentialansatz löst diese Differentialgleichung. Für die
zeitabhängige Stromstärke ergibt sich der Audruck
\begin{equation}
\label{eq:loesung}
I(t) = a \exp({i \omega_1 t}) + b \exp({i \omega_2 t})
\end{equation}
mit
\begin{equation*}
\omega_{1,2} = i \frac{R}{2 L} \pm \sqrt{\frac{1}{LC} - \frac{R^2}{4L^2}}.
\end{equation*}

Wird die Wurzel von $\omega_{1,2}$ reell, so ergibt sich für die Stromstärke \eqref{eq:loesung} eine gedämpfte Schwingung. Die Lösung kann dann umgeformt werden zu
\begin{equation}
\label{eq:loesung_ged}
I(t) = A_0 \exp\left({-\frac{R}{2L}t}\right) \cdot \cos({\omega t + \phi_0})
\end{equation}
Hierbei wird der Exponent $\frac{R}{2L}$ als effektiver Dämpfungswiderstand bezeichnet, $\phi_0$ ist eine konstante Phase und $A_0$ die Anfangsstromstärke, welche durch Anfangsbedingungen festgelegt werden.

Wird $\omega_{1,2}$ aus \eqref{eq:loesung} imaginär, so werden alle Exponenten in Gleichung \eqref{eq:loesung} reell und negativ, weswegen der RLC-Kreis in diesem Fall keine Schwingung, sondern einen exponentiellen Abfall der Stromstärke ausführt.
Nach einigen Umformungen ergibt sich die Stromstärke zu
\begin{equation}
\label{eq:loesung_ged_reell}
I(t) = A_0 \exp\left({-\frac{R}{2L}t}\right).
\end{equation}
Die Stromstärke geht am schnellsten gegen Null, wenn für den \name{ohm}schen Widerstand in der Schaltung gilt:
\begin{equation}
\label{eq:rap}
R_{\text{ap}} = 2\sqrt{\frac{L}{C}}.
\end{equation}
Dieser Fall wird als aperiodischer Grenzfall bezeichnet.

\subsection{Erzwungene Schwingungen}
Wird von außen an den RLC-Kreis in Abb. \ref{fig:rlc-kreis} eine Wechselspannung angelegt, so führt das System erzwungene Schwingungen aus.
Mithilfe der \name{Kirchhoff}schen Regeln wird eine Differentialgleichung für die zeitabhängige Spannung am Kondensator aufgestellt. Diese ist in Formel \eqref{eq:dgl_ged} wiedergegeben.
\begin{equation}
\label{eq:dgl_ged}
LC \frac{\text{d}^2 U_\text{C}}{\text{d}t^2} + RC \frac{\text{d}U_\text{C}}{\text{d}t} + U_\text{C} = U_0 \exp({i\omega t})
\end{equation}
Diese Differentialgleichung wird ebenfalls durch einen Exponentialansatz gelöst. Als kreisfrequenzabhängige Amplitudenspannung $U_\text{C}(\omega)$ ergibt sich
\begin{equation}
\label{eq:amplitude_freq}
U_\text{C}(\omega) = \frac{U_0}{ \sqrt{ (1 - LC \omega^2)^2 + \omega^2 R^2 C^2  }}.
\end{equation}
$U_0$ ist hierbei die Amplitudenspannung der Spannungsquelle.

Die Phase $\phi$ zwischen der Erreger- und Kondensatorspannung ist ebenfalls frequenzabhängig. Diese errechnet sich mit der aus \eqref{eq:dgl_ged} erhaltenen komplexen Lösung U wie folgt:
\begin{equation}
\label{eq:phase_freq}
\phi(\omega) = \arctan{\left(\frac{-\omega R C}{1 - LC \omega^2}\right)}
\end{equation}
