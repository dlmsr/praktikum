% This work is licensed under the Creative Commons
% Attribution-NonCommercial 3.0 Unported License. To view a copy of this
% license, visit http://creativecommons.org/licenses/by-nc/3.0/.

\section{Auswertung}

\subsection{Bestimmung des  effektiven Dämpfungswiderstands}
Zur Bestimmung des effektiven Dämpfungswiderstands werden aus
Abbildung~\ref{fig:thermodruck} die Minima und Maxima abgelesen. In
Tabelle~\ref{tab:minmax-thermo} finden sich diese Werte. Zur Bestimmung
des Exponenten $m := R/(2L)$ in Gleichung~\eqref{eq:loesung} werden die
Maxima und Minima einzeln behandelt. Aus diesem Exponent kann dann
mithilfe von
\begin{equation}
	R = 2mL
	\label{eq:exp-widerstand}
\end{equation}
der effektive Dämpfungswiderstand bestimmt werden.

Zwei Ausgleichsrechnungen werden durchgeführt, da in der
Abbildung~\ref{fig:minmax-plot} zu erkennen ist, daß die Punkte zu zwei
verschiedenen Geraden gehören. Daher werden die Minima und Maxima
getrennt durch eine Gerade
\begin{equation}
	y = \log U = mt + b
\end{equation}
ausgeglichen und der Mittelwert der Steigungen bestimmt. Das Ergebnis
kann in Tabelle~\ref{tab:daempfung-ergebnis} gefunden werden.

\begin{table}
\centering\footnotesize
  \begin{tabular}{lSSS}
    \toprule & {$m/\si{\per\second}$} & {$R_\text{eff}/\si{\ohm}$} &
             {$\frac{\Delta R_\text{eff}}{R}/\si{\percent}$}\\ \midrule
             Minima & 6538 \\ Maxima & 4989 \\ \midrule Mittel & 5764 &
             117 & 314 \\ Standardabweichung & 775 \\ \bottomrule
  \end{tabular}
  \caption{Die Ergebnisse der Ausgleichsrechnung. Der Exponent wird nach
    Formel~\eqref{eq:exp-widerstand} in den Dämpfungswiderstand
    umgerechnet. Die Abweichung wird vom eingebauten Widerstand, dessen
    Wert sich in Tabelle~\ref{tab:r_ap-daten} befindet, berechnet.}
  \label{tab:daempfung-ergebnis}
\end{table}

\begin{table}
  \centering\footnotesize
\begin{tabular}{SSSSSSSS}
\toprule \multicolumn{2}{c}{Maxima} & \multicolumn{2}{c}{Minima} &
\multicolumn{2}{c}{Maxima} & \multicolumn{2}{c}{Minima} \\ \midrule
            {$t/\si{\micro\second}$} & {$U_C/\si{\volt}$} &
            {$t/\si{\micro\second}$} & {$U_C/\si{\volt}$} &
            {$t/\si{\micro\second}$} & {$U_C/\si{\volt}$} &
            {$t/\si{\micro\second}$} & {$U_C/\si{\volt}$} \\ \midrule
            0.0 & 6.4 & 15.0 & 6.4 & 35.0 & 5.2 & 45.0 & 5.2 \\ 60.0 &
            4.4 & 75.0 & 4.8 & 90.0 & 3.6 & 105.0 & 4.0 \\ 120.0 & 2.8 &
            135.0 & 3.2 & 150.0 & 2.4 & 165.0 & 2.8 \\ 180.0 & 2.0 &
            195.0 & 2.4 & 215.0 & 1.6 & 225.0 & 2.4 \\ \bottomrule
\end{tabular}
\caption{Hier sind die aus der Abbildung~\ref{fig:thermodruck}
  entnommenen Punkte der Einhüllenden. Die Punkte sind nach Maxima und
  Minima getrennt. Die Wertepaare $(t, \log U)$ werde in eine lineare
  Ausgleichsrechnung gegeben.}
\label{tab:minmax-thermo}
\end{table}

\begin{figure}[h]
  \centering \includegraphics[width=0.7\textwidth]{exp-plot}
  \caption{Hier sind die aus Abbildung~\ref{fig:thermodruck} abgelesenen
    Werte halblogarithmisch aufgetragen. Die zwei Geraden sind deutlich
    zu erkennen. Die obere Ausgleichsgerade verläuft durch die Minima
    der Amplitude, die untere durch die Maxima. Die beiden Steigungen
    sind dann gemittelt worden.}
  \label{fig:minmax-plot}
\end{figure}  

\begin{figure}[b]
	\centering \includegraphics{bild}
	\caption{Diese Graphik wurde mit dem Oszilloskop aufgenommen.
          Es ist der zeitliche Verlauf der Kondensatorspannung
          aufgezeichnet. Der exponentielle Abfall der Amplitude ist zu
          erkennen.}
	\label{fig:thermodruck}
\end{figure}

\subsection{Bestimmung des aperiodischen Grenzwerts}

In Tabelle~\ref{tab:r_api-daten} ist der experimentell bestimmte Wert
des Grenzwiderstandes $R_\text{ap}$ sowie der berechnete Widerstand mit
den entsprechenden Abweichungen aufgelistet. Darin finden sich auch die
zur Berechnung verwendeten Daten des Versuchsaufbaus.

\begin{table}
  \centering\footnotesize
  \begin{tabular}{SSSSSS}
    \toprule {$L/\si{\milli\henry}$} & {$C/\si{\micro\farad}$} &
             {$R/\si{\ohm}$} & {$R_\text{ap, theo}/\si{\ohm}$} &
             {$R_\text{ap, exp}/\si{\ohm}$} & {$\frac{\Delta
                 R_\text{ap}}{R_\text{ap, theo}}/\si{\percent}$}
             \\ \midrule 10.14 & 2.088 & 54.7 & 139.4 & 3410 & 235
             \\ \bottomrule
  \end{tabular}
  \caption{Für den Versuch ist Gerät~2 verwendet worden. Hier sind die
    entsprechenden Daten und der berechnete Wert für den Widerstand des
    aperiodischen Grenzfalls, sowie der experimentell bestimmte Wert mit
    einer prozentualen Abweichung vom berechneten Wert angegeben.}
  \label{tab:r_ap-daten}
\end{table}


\subsection{Bestimmung der Resonanzüberhöhung}

Die Meßwerte, die in diesem Teil des Versuches aufgenommen worden sind,
finden sich in Tabelle~\ref{tab:resonanz}. In
Abbildung~\ref{fig:resonanz} ist im oberen Plot das Verhältnis $U_C/U_0$
gegen die Frequenz $\nu$ halblogarithmisch aufgetragen und im unteren
Plot ist $U_C/U_0$ im Bereich um die Resonanzfrequenz herum linear gegen
$\nu$ aufgetragen. Dabei wurden alle Meßwerte beachtet, für die die
Kondensatorspannung $U_C > \SI{60}{\volt}$ ist. In der
Tabelle~\ref{tab:resonanz-ergebnis} sind die Ergebnisse für die Güte und
die Breite der Resonanzspitze abzulesen.

\begin{table}
  \centering\footnotesize
  \begin{tabular}{SSSSSSSS}
    \toprule
    {$\nu / \si{\kilo\hertz}$} & {$ U_C/\si{\volt}$} &
    {$\nu / \si{\kilo\hertz}$} & {$ U_C/\si{\volt}$} &
    {$\nu / \si{\kilo\hertz}$} & {$ U_C/\si{\volt}$}\\
    \midrule
    10.0 & 16.0  &  11.0 & 46.0  &  12.0 & 20.5  &  13.0 & 11.5  \\
    14.0 & 10.0  &  15.0 & 10.0  &  16.0 & 10.5  &  17.0 & 11.5  \\
    18.0 & 12.0  &  19.0 & 12.5  &  20.0 & 13.5  &  21.0 & 14.5  \\
    22.0 & 15.5  &  23.0 & 16.5  &  24.0 & 18.0  &  25.0 & 19.5  \\
    26.0 & 22.0  &  27.0 & 25.0  &  28.0 & 29.0  &  29.0 & 34.0  \\
    30.0 & 42.0  &  30.5 & 49.0  &  31.0 & 57.0  &  31.5 & 68.0  \\
    32.0 & 84.0  &  32.5 & 110.0 &  33.0 & 140.0 &  33.5 & 155.0 \\
    34.0 & 140.0 &  34.5 & 110.0 &  35.0 & 86.0  &  35.5 & 79.0  \\
    36.0 & 56.0  &  36.5 & 48.0  &  37.0 & 41.0  &  37.5 & 36.0  \\
    38.0 & 31.0  &  38.5 & 28.5  &  39.0 & 26.0  &  39.5 & 23.5  \\
    40.0 & 21.5  &  41.0 & 18.5  &  42.0 & 16.0  &  43.0 & 14.0  \\
    44.0 & 12.5  &  45.0 & 11.5  &  46.0 & 10.0  &  47.0 & 9.0   \\
    48.0 & 8.5   &  49.0 & 8.0   &  50.0 & 7.5 \\
    \bottomrule
  \end{tabular}
  \caption{Hier finden sich die Werte der Kondensatorspannung in
    Abhängigkeit von der Frequenz gemessen.}
  \label{tab:resonanz}
\end{table}

\begin{figure}[h]
  \centering
  \includegraphics[width=0.7\textwidth]{resonanz}
  \caption{Das Verhältnis $U_C/U_0$ ist gegen $\nu$ einmal
    halblogarithmisch und einmal linear aufgetragen. Beim linearen Plot
    ist nur der Bereich um die Resonanzspitze beachtet worden.}
  \label{fig:resonanz}
\end{figure}

\begin{table}
  \centering\footnotesize
  \begin{tabular}{lSS}
    \toprule
    & {$q$} & {$(\nu_+-\nu_-)/\si{\kilo\hertz}$} \\
    \midrule
    Berechnet  & 1.274 &  5395 \\
    Experiment & 14.09 & 25133 \\
    Abweichung (in \%) & 1006 & 314\\
    \bottomrule
  \end{tabular}
  \caption{Hier die Ergebnisse der Resonanzmessung}
  \label{tab:resonanz-ergebnis}
\end{table}

\begin{figure}[h]
  \centering
  \includegraphics[width=0.7\textwidth]{phasen-plot}
  \caption{Hier ist der Phasenunterschied $\phi$ gegen die Frequenz
    $\nu$ aufgetragen. Im oberen Plot sind alle Meßwerte
    halblogarithmisch und im unteren Plot ist der Bereich um
    \SI{90}{\degree} linear dargestellt.}
  \label{fig:phasen-plot}
\end{figure}
