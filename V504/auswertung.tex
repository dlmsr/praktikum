% This work is licensed under the Creative Commons
% Attribution-NonCommercial 3.0 Unported License. To view a copy of this
% license, visit http://creativecommons.org/licenses/by-nc/3.0/.

\section{Auswertung}

\subsection{Bestimmung der Sättigungsstromstärke aus den Kennlinien}

In den \crefrange{fig:kennlinie-2.1A}{fig:kennlinie-2.6A} auf den
\cpagerefrange{fig:kennlinie-2.1A}{fig:kennlinie-2.6A} sind die
aufgenommenen Kennlinien zu sehen. Die horizontale Linie kennzeichnet
den Sättigungsstrom. In \cref{tab:saettigung} sind die abgelesenen Werte
mit den zugehörigen Heizspannungen und -strömen gelistet.

\begin{figure}
  \centering
  \includegraphics[width=.7\textwidth]{kennlinie-2_1A}
  \caption{Die Abbildung zeigt den Strom-Spannungsverlauf bei einem
    Heizstrom von \SI{2.1}{\ampere}. Die gestrichelte Linie kennzeichnet
  die Sättigungsstromstärke. Raumladungsgebiet und Sättigungsbereich
  sind nicht sehr deutlich voneinander zu unterscheiden.}
  \label{fig:kennlinie-2.1A}
\end{figure}

\begin{figure}
  \centering
  \includegraphics[width=.7\textwidth]{kennlinie-2_2A}
  \caption{Die Abbildung zeigt den Strom-Spannungsverlauf bei einem
    Heizstrom von \SI{2.2}{\ampere}. Die gestrichelte Linie kennzeichnet
    die Sättigungsstromstärke. Raumladungsgebiet und Sättigungsbereich
    sind nicht sehr deutlich voneinander zu unterscheiden.}
  \label{fig:kennlinie-2.2A}
\end{figure}

\begin{figure}
  \centering
  \includegraphics[width=.7\textwidth]{kennlinie-2_3A}
  \caption{Die Abbildung zeigt den Strom-Spannungsverlauf bei einem
    Heizstrom von \SI{2.3}{\ampere}. Die gestrichelte Linie kennzeichnet
    die Sättigungsstromstärke. Raumladungsgebiet und Sättigungsbereich
    sind nicht sehr deutlich voneinander zu unterscheiden.}
  \label{fig:kennlinie-2.3A}
\end{figure}

\begin{figure}
  \centering
  \includegraphics[width=.7\textwidth]{kennlinie-2_4A}
  \caption{Die Abbildung zeigt den Strom-Spannungsverlauf bei einem
    Heizstrom von \SI{2.4}{\ampere}. Die gestrichelte Linie kennzeichnet
    die Sättigungsstromstärke. Raumladungsgebiet und Sättigungsbereich
    sind nicht sehr deutlich voneinander zu unterscheiden.}
  \label{fig:kennlinie-2.4A}
\end{figure}

\begin{figure}
  \centering
  \includegraphics[width=.7\textwidth]{kennlinie-2_6A}
  \caption{Die Abbildung zeigt den Strom-Spannungsverlauf bei einem
    Heizstrom von \SI{2.6}{\ampere}. Die gestrichelte Linie kennzeichnet
    die Sättigungsstromstärke. Raumladungsgebiet und Sättigungsbereich
    sind nicht sehr deutlich voneinander zu unterscheiden.}
  \label{fig:kennlinie-2.6A}
\end{figure}

\begin{table}
  \centering
  \begin{tabular}{rSSS}
    \toprule
    & {} & {} & \\
    \midrule
    \\
    \bottomrule
  \end{tabular}
  \caption{Aus den \crefrange{fig:kennlinie-2.1A}{fig:kennlinie-2.6A}
    abgelesene Sättigungsstromstärken. Es ist zu erkennen, daß der
    Sättigungsstrom sehr stark mit der Temperatur ansteigt.}
  \label{tab:saettigung}
\end{table}
