% This work is licensed under the Creative Commons
% Attribution-NonCommercial 3.0 Unported License. To view a copy of this
% license, visit http://creativecommons.org/licenses/by-nc/3.0/.

\section{Auswertung}

\subsection{Ohne Rauschen}

Gemessen wird die Ausgangsgleichspannung $U_\text{out}$ in Abhängigkeit
der Phase $\phi$. Die erhaltenen Werte finden sich in
Tabelle~\ref{tab:messwerte}. Diese werden nun in eine nichtlineare
Regression der Form
%
\begin{equation}
  \label{eq:cos-regress}
  U_\text{out}(\phi) = A\cos(\phi+\phi_0)
\end{equation}
%
gegeben und mit dem theoretischen Ergebnis nach
Formel~\eqref{eq:gleichspannung} verglichen. Für die nichtlineare
Regression wird die
\texttt{scipy}-Bibliothek\footnote{\url{http://www.scipy.org/}} in der
Version 0.11.0 verwendet. Das Ergebnis ist in
Abbildung~\ref{fig:ohne-rauschen} und hier zu sehen:
%
\begin{equation}
  \label{eq:cos-regress-result}
  U_\text{out}(\phi) = \SI{-85.13(384)}{\milli\volt} \cdot
  \cos(\phi + \num{0.20})
\end{equation}

\subsection{Mit Rauschen}

Nun wird das mit Rauschen überlagerte Signal untersucht. Die Meßwerte
können ebenfalls in Tabelle~\ref{tab:messwerte} gefunden werden. Für die
nichtlineare Regression mit Formel~\eqref{eq:cos-regress} wird wieder
die \texttt{scipy}-Bibliothek verwendet. In
Abbildung~\ref{fig:mit-rauschen} kann man die Regression sehen. Die
Gleichung dazu lautet:
%
\begin{equation}
  \label{eq:cos-regress-result-noise}
  U_\text{out}(\phi) = \SI{2443(76)}{\milli\volt} \cdot 
  \cos(\phi + \num{1.29})
\end{equation}

\begin{table}
  \centering
  \begin{tabular}{SSSS}
    \toprule
    \multicolumn{2}{c}{Mit Rauschen} &
    \multicolumn{2}{c}{Ohne Rauschen} \\
    \midrule
    {\(\phi [\si{\degree}]\)} & 
    {\(U_\text{out} [\si{\milli\volt}]\)} &
    {\(\phi [\si{\degree}]\)} & 
    {\(U_\text{out} [\si{\milli\volt}]\)} \\
    \midrule
    0   &   504  &   0 & -80.0 \\
    15  &   280  &  15 & -78.0 \\
    30  &  -424  &  30 & -72.0 \\
    75  & -2240  &  45 & -52.8 \\
    45  & -1080  &  60 & -18.4 \\
    60  & -1840  &  75 &   5.6 \\
    90  & -2360  &  90 &  17.6 \\
    105 & -2400  & 105 &  26.8 \\
    180 &  -536  & 180 &  84.0 \\
    270 &  2320  & 270 & -17.6 \\
    \bottomrule
  \end{tabular}
  \caption{Meßwerte der Gleichspannung $U_\text{out}$ in Abhängigkeit der
    Phase $\phi$ des Referenzsignals}
  \label{tab:messwerte}
\end{table}
