% This work is licensed under the Creative Commons
% Attribution-NonCommercial 3.0 Unported License. To view a copy of this
% license, visit http://creativecommons.org/licenses/by-nc/3.0/.

\section{Versuchsaufbau}

Da verschiedene Eigenschaften der Mikrowellen und des Klystron
untersucht werden, ist der Aufbau dieses Versuchs flexibel.  Je nach
Versuchsteil müssen verschiedene Elemente aufgebaut werden.  Der
grundlegende Aufbau bleibt aber derselbe und ist in \cref{fig:aufbau}
dargestell.

Links steht das Klystron, das für die Erzeugung der Mikrowellen
zuständig ist, dicht gefolgt von einem Isolator, dem Frequenzzähler,
einem Dämpfungsglied und einem Detektor am Ende. Je nach Versuchsteil
befinden sich zwischen Detektor und Dämpfungsglied verschiedene andere
Bauteile oder der Detektor entfällt ganz und wird durch verschiedene
Abschlüsse ersetzt.  Zum Darstellen des Signals, das vom Detektor oder
verschiedenen Sonden aufgenommen wird, werden zwei Meßgeräte verwendet:
Ein sogenanntes \emph{Standing Wave Ratio (SWR)} Meter und ein
Oszilloskop.

\section{Durchführung}
