% This work is licensed under the Creative Commons
% Attribution-NonCommercial 3.0 Unported License. To view a copy of this
% license, visit http://creativecommons.org/licenses/by-nc/3.0/.

\section{Theorie}

\subsection{Erzeugung von Mikrowellen mit dem Klystron}

Es gibt verschiedene Wege Mikrowellenstrahlung zu erzeugen.  In diesem
Versuch wird ein sogenannten Reflex-Klystron verwendet.  Ein Klystron
ist eine Mikrowellenröhre, in der aus einem Elektronenstrahl durch
Geschwindigkeitsmodulation Energie in Form von Mikrowellen gewonnen
wird.  Der wesentliche Aufbau des Klystrons ist in \cref{fig:klystron}
skizziert.  Es besteht aus drei grundlegenden Komponenten: beheizte
Kathode, Resonator mit Gitter, Reflektorelektrode.

Elektronen treten aufgrund des \name{Edison}-Effekts aus der beheizten
Kathode aus und werden zum Reflektor beschleunigt. Dabei laufen sie
durch ein Gitter und werden am Reflektor, der gegenüber der Kathode auf
einem niedrigeren Potential liegt, wieder reflektiert (daher der Name
Reflex-Klystron).  Die Beschleunigung übernimmt der Resonator, der auf
einem höheren Potential als die Kathode liegt.

Wenn das Klystron nun schwingt, befindet sich im Resonator zwischen den
Gittern ein elektromagnetisches Feld.  Je nach Feldverteilung am Gitter
werden die Elektronen beschleunigt oder verzögert.  Dadurch sammeln sie
sich wegen ihres dabei entstehenden Geschwindigkeitsunterschieds in
Bündel.  Diese Form der Geschwindigkeitsänderung heißt
Geschwindigkeitsmodulation.  Die Bündel, die durch den Resonator laufen,
treten in Wechselwirkung mit dem zwischen den Gittern vorhandenen Feld.
Werden sie dort abgebremst, so geben sie Energie an den Resonator ab;
werden sie beschleunigt, nehmen sie dem Resonator Energie ab.

Da die Frequenz des Klystrons durch die Abmessungen des
Resonatorhohlraumes bestimmt sind, kann durch eine Änderung des
Resonatorvolumens die Frequenz des Klystrons geändert werden
(mechanische Abstimmung).  Man spricht dabei vom Stimmen des Klystrons.
Kleinere Frequenzänderungen können auch durch Abgleich der Reflektor-
oder Resonatorspannung erzielt werden (elektronische Abstimmung).  Es
ist auch eine Frequenz- oder Amplitudenmodulation des Klystrons über die
Modulation der Reflektorspannung möglich.


\subsection{Führung der Mikrowellen auf Hohlleitern}

Hohlleiter führen elektromagnetische Wellen und transportieren über
diese Energie von einem Punkt zu einem anderen.  Als Hohlleiter kann
jedes Rohr aus einem leitenden Material benutzt werden.  Häufig
vorkommende Geometrien sind allerdings rechteckige oder kreisförmige
Querschnitte, seltener elliptische.  In diesem Versuch werden
rechteckige Hohlleiter verwendet.  Abhängig von der Geometrie,
d.\,h. insbesondere den Abmessungen des Hohlleiters, ergibt sich eine
Grenzfrequenz, unterhalb derer keine Welle geleitet und entsprechend
keine Leistung transportiert werden kann 

Ein Hohlleiter kann verschiedene Formen von WM-Wellen leiten, die Moden
genannt werden. Jeder Modus hat seine eigene Feldverteilung und
Grenzfrequenz. Es gibt dabei zwei grundlegende Arten von Moden:
transversal elektrische Moden und transversal magnetische Moden.  Das
elektrische Feld eines transversal elektrischen Modus (TE-Modus) ist
überall senkrecht zur Ausbreitungsrichtung im Hohlleiter.  Ebenso hat
ein transversal magnetischer Modus (TM-Modus) ein magnetisches Feld, das
überall senkrecht zur Ausbreitungsrichtung ist.

% TODO: Kennzeichnung der Moden

% TODO: Wellenlänge, Frequenz und Dämpfung
