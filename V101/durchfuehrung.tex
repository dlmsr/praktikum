% This work is licensed under the Creative Commons
% Attribution-NonCommercial 3.0 Unported License.  To view a copy of
% this license, visit http://creativecommons.org/licenses/by-nc/3.0/.

\section{Aufbau und Durchführung}

\subsection{Aufbau der Apparatur}

Zur Untersuchung der Trägheitsmomente wird in diesem Versuch eine
sogenannte Drillachse verwendet.  In \cref{fig:drillachse} ist eine
solche skizziert.  Sie besteht aus einer drehbar gelagerten Achse, die
über eine Spiralfeder an den Rahmen der Lagerung oszillatorisch
gekoppelt ist.  An der Spitze dieser Achse werden die zu untersuchenden
Körper befestigt.

Mithilfe dieser Apparatur können die eingespannten Körper nun zu
Schwingungen angeregt werden, über deren Schwingungsdauer sich dann das
Trägheitsmoment derselben berechnen läßt.  Dabei muß beachtet werden,
daß in Formel~\eqref{eq:periode} das Gesamtträgheitsmoment steht.  Im
Falle dieser Versuchsanordnung setzt sich dieses aus dem
Eigenträgheitsmoment~$I_\text{D}$ der Drillachse und dem Trägheitsmoment
$I_\text{K}$ des zu untersuchenden Körpers zusammen.  Sind also
Richtgröße~$D$ und Eigenträgheitsmoment~$I_\text{D}$ der Apparatur
bekannt, so kann das Trägheitsmoment~$I_\text{K}$ gemäß
\begin{equation}
  \label{eq:traegheit-winkelricht-drill}
  I_\text{K} = \frac{D T^2}{4 \pi^2} - I_\text{D}
\end{equation}
bestimmt werden.

\begin{figure}
  \centering
  \includegraphics[height=7cm]{drillachse}
  \caption{Skizze der verwendeten Versuchsapparatur. Zu sehen ist die
    Drillachse, die drehbar in der Halterung gelagert ist. An der
    Halterung ist eine Spiralfeder befestigt, die an die Drillachse
    gekoppelt ist und die Halterung für die Befestigung der Proben.}
  \label{fig:drillachse}
\end{figure}

\subsection{Bestimmung der Winkelrichtgröße}

Zur Bestimmung der Winkelrichtgröße~$D$ wird ein dünner Stab an der
Drillachse so befestigt, daß die Drehachse durch den Schwerpunkt
verläuft.  Jetzt wird in einem Abstand~$r$ vom Drehpunkt eine Federwaage
am Stab eingehängt und um den Winkel~$\phi$ ausgelenkt.  Hierbei sollte
die Federwaage am besten senkrecht zum Stab und zur Rotationsachse
gehalten werden, da so der Betrag vom $\vec{M}$ einfach durch
\begin{equation}
  | \vec{M} | = | \vec{r} \times \vec{F} | = r F \sin \angle(\vec{r},
  \vec{F}) = r F
\end{equation}
bestimmt werden kann.  Die Winkelrichtgröße kann dann gemäß
Gleichung~\eqref{eq:drehmoment-winkelricht} so bestimmt werden:
\begin{equation}
  \label{eq:winkelricht-kraft-abstand-winkel}
  D = \frac{r F}{\phi}.
\end{equation}
Hierbei wird sowohl der Abstand als auch der Winkel variert, um eine
Meßreihe mit zehn Meßwerten zu erhalten.

\subsection{Bestimmung des Trägheitsmoments der Drillachse}

Das Trägheitsmoment der Drillachse wird ermittelt, indem ein nahezu
masselose Stab, mit welchem zwei Gewichte in gleichem Abstand~$a$ vom
Schwerpunkt starr verbunden sind, an der Drillachse befestigt wird.  Die
Trägheitsmomente der Gewichte sowie des Stabes sind bekannt.  Das
beschriebene System wird nun für verschiedene Abstände~$a$ zu
Schwingungen angeregt und die Schwingungsdauer~$T$ wird mit einer
Stoppuhr bestimmt.  Das Gesamtträgheitsmoment in
Formel~\eqref{eq:periode} setzt sich also so zusammen:
\begin{equation}
  I = I_\text{D} + I_\text{S, G}
\end{equation}
Hier bezeichnet $I_\text{S, G}$ das Gesamtträgheitsmoment des Stabes und
der Gewichte.  Nach dem \name{Steiner}schen Satz ergibt sich weiter:
\begin{equation}
  \label{eq:gesamttraegheitsmoment}
  I = I_\text{D} + \left(I_\text{S} + 2I_\text{G} +  2m a^2\right),
\end{equation}
wobei $I_\text{S}$ das Trägheitsmoment des verwendeten Stabes und
$I_\text{G}$ das Trägheitsmoment eines Gewichts bei Rotation um eine
Achse durch den Schwerpunkt parallel zur Rotationsachse der Drillachse
bezeichnet.  Bei der angegebenen Masse handelt es sich um die Masse der
Gewichte, die als gleich schwer angenommen werden.  Die Gleichung, die
sich durch Einsetzen von $I$ in Gleichung~\eqref{eq:periode} ergibt,
läßt sich so umstellen:
\begin{equation}
  \label{eq:regressionsform}
  T^2 = \frac{8\pi^2 m}{D} a^2 + \frac{4\pi^2}{D}\big(I_\text{D} 
  + I_\text{S} + 2I_\text{G}\big)
\end{equation}
Es wird eine Meßreihe der Schwingungsdauer für zehn verschiedene
Abstände~$a$ des Gewichts angefertigt.  Aus
Gleichung~\eqref{eq:regressionsform} geht hervor, daß hieraus kann durch
eine lineare Regression mit der Methode der kleinsten Quadrate das
Eigenträgheitsmoment aus dem Ordinatenabschnitt bestimmt werden.

\subsection{Untersuchung des Trägheitsmoments}

Im zweiten Versuchsteil wird das Trägheitsmoment zweier verschiedener
Körper sowie dasjenige einer Modellpuppe in zwei verschiedenen Haltungen
untersucht werden.  Die Messung läuft im Prinzip analog zur Bestimmung
des Trägheitsmoments der Drillachse: Das zu untersuchende Objekt wird
mit der Drillachse starr verbunden und dann zur Schwingung gebracht.
Aus der Bestimmung der Schwingungsdauer durch eine Stoppuhr kann
mithilfe der zuvor bestimmten Apparaturkonstanten das Trägheitsmoment
ermittelt werden.  Hier wird jeweils eine Meßreihe mit zehn Werten
aufgenommen, wobei so verfahren wird, daß beide Versuchsteilnehmer zu
einer Schwingung mit zwei Stoppuhren die Zeit nehmen.

Am Ende werden die erhaltenen Ergebnisse aus der Bestimmung des
Trägheitsmoments mit theoretisch ermittelten Werten verglichen.  Diese
Werte werden durch eine Näherung der Modellpuppe durch Zylinder
bestimmt.
