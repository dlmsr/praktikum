% This work is licensed under the Creative Commons
% Attribution-NonCommercial 3.0 Unported License.  To view a copy of
% this license, visit http://creativecommons.org/licenses/by-nc/3.0/.

\section{Aufbau und Durchführung}

\subsection{Aufbau der Apparatur}

Zur Untersuchung der Trägheitsmomente wird in diesem Versuch eine
sogenannte Drillachse verwendet.  In \cref{fig:drillachse} ist eine
solche skizziert.  Sie besteht aus einer drehbar gelagerten Achse, die
über eine Spiralfeder an den Rahmen der Lagerung oszillatorisch
gekoppelt ist.  An der Spitze dieser Achse werden die zu untersuchenden
Körper befestigt.

Mithilfe dieser Apparatur können die eingespannten Körper nun zu
Schwingungen angeregt werden, über deren Periodendauer sich dann das
Trägheitsmoment derselben berechnen läßt.  Dabei muß beachtet werden,
daß in Formel~\eqref{eq:periode} das Gesamtträgheitsmoment steht.  Im
Falle dieser Versuchsanordnung setzt sich dieses aus dem
Eigenträgheitsmoment~$I_\text{D}$ der Drillachse und dem Trägheitsmoment
$I_\text{K}$ des zu untersuchenden Körpers zusammen.  Sind also
Richtgröße~$D$ und Eigenträgheitsmoment~$I_\text{D}$ der Apparatur
bekannt, so kann das Trägheitsmoment~$I_\text{K}$ gemäß
\begin{equation}
  \label{eq:traegheit-winkelricht-drill}
  I_\text{K} = \frac{D T^2}{4 \pi^2} - I_\text{D}
\end{equation}
bestimmt werden.

\subsection{Bestimmung der Winkelrichtgröße}

Zur Bestimmung der Winkelrichtgröße~$D$ wird ein dünner Stab an der
Drillachse so befestigt, daß die Drehachse durch den Schwerpunkt
verläuft.  Jetzt wird in einem Abstand~$r$ vom Drehpunkt eine Federwaage
am Stab eingehängt und um den Winkel~$\phi$ ausgelenkt.  Hierbei sollte
die Federwaage am besten senkrecht zum Stab und zur Rotationsachse
gehalten werden, da so der Betrag vom $\vec{M}$ einfach durch
\begin{equation}
  | \vec{M} | = | \vec{r} \times \vec{F} | = r F \sin \angle(\vec{r},
  \vec{F}) = r F
\end{equation}
bestimmt werden kann.  Die Winkelrichtgröße kann dann gemäß
Gleichung~\eqref{eq:drehmoment-winkelricht} so bestimmt werden:
\begin{equation}
  \label{eq:winkelricht-kraft-abstand-winkel}
  D = \frac{r F}{\phi}.
\end{equation}
Hierbei wird sowohl der Abstand als auch der Winkel variert, um eine
Meßreihe mit zehn Meßwerten zu erhalten.

\subsection{Bestimmung des Trägheitsmoments der Drillachse}

Das Trägheitsmoment der Drillachse wird ermittelt, indem eine nahezu
masselose Stange, mit der zwei Gewichte in gleichem Abstand~$a$ vom
Schwerpunkt starr verbunden sind, an der Drillachse befestigt wird.  Die
Trägheitsmomente der Gewichte sowie der Stange sind bekannt.  Das
beschriebene System wird nun für verschiedene Abstände~$a$ zu
Schwingungen angeregt und die Periodendauer~$T$ wird bestimmt.  Das
Gesamtträgheitsmoment in Formel~\eqref{eq:periode} setzt sich nun nach
Anwendung des \name{Steiner}schen Satzes für die beiden Gewichte so
zusammen:
\begin{equation}
  I = I_\text{D} + I_\text{S} + 2I_\text{G} + 2m_\text{G} a^2
\end{equation}
Hier bezeichnet $I_\text{S}$ das Trägheitsmoment der Stange und
$I_\text{G}$ dasjenige eines Gewichts bezogen auf einer zur Drehachse
parallen Achse durch den Schwerpunkt.  Da die verwendeten Gewichte
zylinderförmig sind und der Stab, an dem diese befestigt ist, lang und
dünn ist, ergibt sich schließlich für das Trägheitsmoment der Drillachse
aus Formel~\eqref{eq:periode}:
\begin{equation}
  \label{eq:traegheit-drillachse}
  I_\text{D} = \frac{D T^2}{4 \pi^2} - \frac{1}{12} m_\text{S} l^2 -
  m_\text{G} (R^2 - 2 a^2),
\end{equation}
wobei $l$ die Länge des verwendeten Stabes ist und $R$ der halbe
Durchmesser eines Zylinders.  Bei den angegeben Massen handelt es sich
jeweils um die Massen eines der Gewichte bzw. des Stabes.  Es wird eine
Meßreihe der Periodendauer für zehn verschiedene Abstände~$a$ des
Gewichts angefertigt.

\subsection{Untersuchung des Trägheitsmoments}

