% This work is licensed under the Creative Commons
% Attribution-NonCommercial 3.0 Unported License. To view a copy of this
% license, visit http://creativecommons.org/licenses/by-nc/3.0/.

\section{Auswertung}

\subsection{Bestimmung des Emissionsvermögens}

Aus der gemessenen Thermospannung und Temperatur wird mithilfe einer
linearen Regressionsrechnung das jeweilige Emissionsvermögen bestimmt.

Da sowohl vor und nach der Strahlungsmessung keine Offsetspannung
gemessen worden ist, werden die Paare $(T^4, U)$ der Meßdaten aus
Tabelle~\ref{tab:strahlung} in eine lineare Ausgleichsrechnung
gegeben. Hierzu wird die \texttt{numpy}-Bibliothek in der Version
\texttt{0.12} verwendet. Die Ergebnisse sind in
Tabelle~\ref{tab:lin-regress} zu finden.

Die gemessene Thermospannung ist proportional zur abgegebenen
Leisung.
\begin{equation}
  \label{eq:proport}
  U = k P = k \epsilon \sigma T^4 = m T^4,
\end{equation}
wobei $m$ die Steigung der Ausgleichsgeraden bezeichnet. Die schwarz
lackierte Oberfläche wird dabei als ein Schwarzer Strahler betrachtet,
d.\,h. es wird ein Emissionsvermögen $\epsilon = 1$ angenommen. Aus
Formel~\ref{eq:proport} ergibt sich damit für den
Proportionalitätsfaktor $k = m_0/\sigma$, wenn die Steigung der
Ausgleichgerade zur schwarzen Oberfläche mit $m_0$ bezeichnet
wird. Daraus kann das Emissionsvermögen der einzelnen Oberflächen
errechnet werden:
\begin{equation}
  \epsilon = \frac{m}{k \sigma} = \frac{m}{m_0}
\end{equation}
wobei mit $m$ die Steigung der jeweiligen Regressionsgeraden gemeint
ist. Aufgrund der Tatsache, daß die Steigungen fehlerbehaftete Größen
sind, wird eine \name{Gauß}sche Fehlerfortpflanzung für $\epsilon$
durchgeführt.
\begin{equation}
  \label{eq:gauss-epsilon}
  \Delta\epsilon = \sqrt{ \left(\frac{\Delta m}{m_0}\right)^2 +
    \left(\frac{m \Delta m_0}{m_0^2}\right)^2}
\end{equation}

\subsection{Auswertung des Abstandsverhaltens}

Bei der Untersuchung des Abstandsverhaltens ergeben sich die in
Tabelle~\ref{tab:abstand} dargestellten Werte. In
Abbildung~\ref{fig:abstand} ist ein Plot zu sehen, der die aufgenommenen
Meßwerte mit einer nichtlinearen Regression zeigt.
