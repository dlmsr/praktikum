\section{Diskussion}

An der Abbildung~\ref{fig:plot-emission} ist gut zu erkennen, welche
Ergebnisse aussagekräftig sind. Die Vermessung der schwarzen und weißen
Oberfläche ist gut gelungen und liegt im erwarteten Bereich. Bei der
Messung der Thermospannungen der anderen beiden Oberflächen fällt schon
an der Streuung der Meßpunkte auf, daß hier ein großer Störeinfluß
vorliegt. Die errechneten Emissionsvermögen sind verglichen mit dem der
weißen Oberfläche deutlich kleiner. Daher überwiegt hier die Störung
durch äußere Effekte.

Obwohl die Abstandsmessung in beiden Fällen ein quadratisches Verhalten
im Leistungsabfall zeigt, ist die Messung nur von begrenzter
Aussagekraft, da nicht genügend Meßwerte in größerer Entfernung der
Strahlungsquelle aufgenommen werden konnten. Der relative Fehler des
Parameters $a$ beträgt \SI{5.4}{\percent} bei $T = \SI{323.16}{\kelvin}$
bzw. \SI{8.2}{\percent} bei $T = \SI{366.16}{\kelvin}$.
