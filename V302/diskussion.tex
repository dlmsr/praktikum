% This work is licensed under the Creative Commons
% Attribution-NonCommercial 3.0 Unported License. To view a copy of this
% license, visit http://creativecommons.org/licenses/by-nc/3.0/.

\section{Diskussion}
Zum Schluss werden die Ergebnisse dieses Versuches beurteilt.
Die Nullmessungsmethode alles Brückenschaltungen hat sich als äußerst präzise erwiesen, was sich an den kleinen Standardabweichungen der Mittelwerte erkennen lässt.

Der verwendete Funktionsgenerator besitzt einen geringen Kliffaktor von k = \SI{0.47}{\percent}, weswegen die Messwerte bei der \name{Wien}-\name{Robinson}-Brücke beinahe mit der Theoriekurve übereinstimmen. Es ist deutlich zu erkennen, dass die Brückenspannung niemals komplett auf \SI{0}{\volt} fällt. Dies liegt daran, dass der Funktionsgenerator schwache Oberschwingungen miterzeugt, was man daran erkennt, dass der Klirrfaktor größer als Null ist.