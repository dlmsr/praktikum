% This work is licensed under the Creative Commons
% Attribution-NonCommercial 3.0 Unported License. To view a copy of this
% license, visit http://creativecommons.org/licenses/by-nc/3.0/.

\section{Theorie}
\label{sec:theorie}

\subsection{Die allgemeine Brückenschaltung}
\label{sec:allg-bruecke}

Eine Brückenschaltung hat im allgemeinen den in
Abbildung~\ref{fig:allg-bruecke} dargestellten Aufbau.  Die Spannung
zwischen den Punkten $A$ und $B$ ergibt sich, wenn die magnetische
Flußänderung außerhalb der Bauteile als Null angenommen wird, mithilfe
der Maschen- und Knotenregeln von \name{Kirchhoff}. Sie ergeben sich aus
den \name{Maxwell}-Gleichungen und lauten:
%
\begin{align*}
  \sum^n_{k = 1} I_k &= 0 \qquad \text{(Knotenregel)}\\
  \sum^n_{k = 1} U_k &= 0 \qquad \text{(Maschenregel)}
\end{align*}
%
Aus den Knoten- und Maschenregeln ergeben sich für die allgemeine
Brückenschaltung dann die Gleichungen
%
\begin{align}
  I_1 &= I_2, &
  I_3 &= I_4,\\
  U &= -R_1 I_1 + R_3 I_3, &
  -U &= -R_2 I_2 + R_4 I_4.
\end{align}
%
Wenn $U_\text{S}$ die Speisespannung bezeichnet, ist die
Brückenspannung~$U$ gegeben durch
%
\begin{equation}
  U = \frac{R_2 R_3 - R_1 R_4}{(R_3 + R_4)(R_1 + R_2)} U_\text{S}.
\end{equation}
%
Die Brückenspannung verschwindet daher bei beliebiger Speisespannung,
falls
%
\begin{equation}
  \label{eq:abgleich}
  R_2 R_3 = R_1 R_4.
\end{equation}

Sind in der Brückenschaltung auch Kapazitäten und Induktivitäten
verbaut, so können die obigen Widerstände als komplexe Zahlen $Z = X +
\iu Y$ aufgefasst werden und mit der komplexen Wechselstromrechnung
ergibt sich
%
\begin{equation}
  \label{eq:abgleich-komplex}
  Z_2 Z_3 = Z_1 Z_4.
\end{equation}
%
Weil komplexe Zahlen Paare reeller Zahlen sind, müssen für die
Gleichheit also die zwei Bedingungen
%
\begin{gather}
  X_1 X_4 - Y_1 Y_4 = X_2 X_3 - Y_2 Y_3\\
  \intertext{und}
  X_1 Y_4 + X_4 Y_1 = X_2 Y_3 + X_3 Y_2
\end{gather}
%
erfüllt sein.  Daher werden bei einer Wechselstrombrücke zwei
voneinander unabhängige Veränderliche eingebaut sein.

\begin{figure}
  \centering
  \includegraphics[height=5cm]{allg-bruecke}
  \caption{Das Schaltbild der allgemeinen Brückenschaltung. Die
    sogenannte Brückenspannung wird zwischen den Punkten $A$ und $B$
    abgegriffen. Das Schaltbild stammt aus der Anleitung zum
    \textcite{v302}.}
  \label{fig:allg-bruecke}
\end{figure}

\subsection{\name{Wheatstone}-Messbrücke}

In der \name{Wheatstone}schen Brückenschaltung sind nur \name{Ohm}sche
Widerstände verbaut, weswegen die Schaltung sowohl mit Gleich- als auch
mit Wechselstrom betrieben werden kann. Dieser Brücke kann zur
Bestimmung eines unbekannten Widerstandes verwendet werden. Wenn zum
Beispiel $R_1$ in Abbildung~\ref{fig:allg-bruecke} unbekannt ist, wird
aus der Abgleichbedingung~\eqref{eq:abgleich}
%
\begin{equation}
  \label{eq:abgleich-ohmsch}
  R_1 = R_2 \frac{R_3}{R_4}
\end{equation}
%
erhalten.  Da es bei den Widerständen $R_3$ und $R_4$ nur auf das
Verhältnis ankommt, können diese beiden als Potentiometer eingebaut
werden.

\subsection{Kapazitätsmessbrücke} 

Ein realer Kondensator wird durch eine ideale Kapazität und einen
\name{Ohm}schen Widerstand in Reihe modelliert, da es immer zu
sogenannten dielektrischen Verlusten kommt, bei denen ein Teil der
gespeicherten elektrischen Energie in Wärme umgewandelt wird. Für den
komplexen Wechselstromwiderstand folgt daher
\begin{equation} Z = R - \frac{\iu}{\omega C}.
\end{equation} Wie im Abschnitt~\ref{sec:allg-bruecke} über
Wechselstromwiderstände gesagt worden ist, wird in der Schaltung eine
weitere Veränderliche benötigt, um den zusätzlichen Freiheitsgrad zu
berücksichtigen. Dazu wird der Widerstand $R_2$ in
Abbildung~\ref{fig:kap-bruecke} veränderlich sein. Die
Abgleichbedingung~\eqref{eq:abgleich-komplex} liefert dann
%
\begin{gather}
  \label{eq:abgleich-kap}
  R_\text{x} = R_2 \frac{R_3}{R_4}\\
  \intertext{und} 
  C_\text{x} = C_2 \frac{R_4}{R_3}.
\end{gather}

\begin{figure}
  \centering
  \includegraphics[height=5cm]{kap-bruecke}
  \caption{Das Schaltbild der Kapazitätsmessbrücke entnommen aus
    \textcite{v302}. Der Widerstand $R_2$ ist notwendig, weil die Brücke
    durch die Kapazität einen zusätzlichen Freiheitsgrad hat.}
  \label{fig:kap-bruecke}
\end{figure}

\subsection{Induktivitätsmessbrücke} 

Wie ein realer Kondensator durch eine ideale Kapazität und einen
\name{Ohm}schen Widerstand modelliert wird, kann eine Spule durch eine
ideale Induktivität und einen \name{Ohm}schen Widerstand dargestellt
werden (siehe Abbildung~\ref{fig:ind-maxwell-bruecke}), da auch sie
einen Teil der magnetischen Energie in Wärme umwandelt. Also ist
%
\begin{equation}
  Z = R + \iu \omega L
\end{equation}
%
der zugehörige komplexe Wechselstromwiderstand. Ähnlich wie bei der
Kapazitätsmeßbrücke ergeben sich die Abgleichbedingungen
%
\begin{gather}
  \label{eq:abgleich-ind-bruecke} R_\text{x} = R_2 \frac{R_3}{R_4},\\
L_\text{x} = L_2 \frac{R_3}{R_4}
\end{gather}

\begin{figure}
  \centering
  \includegraphics[height=5cm]{ind-bruecke} \qquad
  \includegraphics[height=5cm]{maxwell-bruecke}
  \caption{Die Schaltbilder der Induktivitätsmessbrücke (links) und der
    \name{Maxwell}-Brücke (rechts) entnommen aus \textcite{v302}. Der
    Widerstand $R_2$ in der Induktivitätsmessbrücke ist aus den gleichen
    Gründen wie bei der Kapazitätsbrücke variabel, um den zusätzlichen
    Freiheitsgrad durch die Induktivität zu kompensieren.  In der
    \name{Maxwell}-Brücke sind die Widerstände $R_3$ und $R_4$
    variabel. Mit diesen beiden Brücken können daher Induktivitäten
    bestimmt werden.}
  \label{fig:ind-maxwell-bruecke}
\end{figure}

\subsection{\name{Maxwell}-Brücke} 

Abbildung~\ref{fig:ind-maxwell-bruecke} zeigt die sogenannte
\name{Maxwell}-Brücke. Die Widerstände $R_3$ und $R_4$ sind regelbar.
Die komplexen Wechselstromwiderstände sind also
%
\begin{gather} Z_1 = R_\text{x} + \iu \omega L_\text{x}\\ \frac{1}{Z_4}
= \frac{1}{R_4} + \iu \omega C_4.
\end{gather}
%
Mit der Abgleichbedingung~\eqref{eq:abgleich-komplex} kann dann
%
\begin{gather}
  \label{eq:abgleich-ind-maxwell} R_\text{x} = R_2 \frac{R_3}{R_4},\\
L_\text{x} = R_2 R_3 C_4
\end{gather}
%
gefunden werden.

\subsection{\name{Wien}-\name{Robinson}-Brücke}

Bis jetzt sind die beschriebenen Brücken zumindest der Theorie nach
frequenzunabhängig. Die \name{Wien}-\name{Robinson}-Brücke hingegen ist
eine frequenzabhängige Brücke. Die Schaltung ist in
Abbildung~\ref{fig:wien-robinson-bruecke} zu sehen. Das entscheidende an
dieser Brücke wird deutlich, wenn das Verhältnis von Brücken- und
Speisespannung betrachtet wird:
%
\begin{equation} 
  \left|\frac{U_\text{Br}}{U_\text{S}}\right|^2 =
  \frac{(1 - (\omega R C)^2)^2}{9(1 - (\omega R C)^2)^2 + 81 (\omega R
    C)^2}.
\end{equation}
%
Hieran kann gesehen werden, dass die Brückenspannung verschwindet, falls
%
\begin{equation} 
  \omega = \omega_0 := \frac{1}{RC},
\end{equation} 
%
Mit $\Omega := \omega/\omega_0$ kann das Spannungsverhältnis
umgeschrieben werden, nämlich zu
%
\begin{equation}
  \label{eq:wien_robinson_theo}
  \left|\frac{U_\text{Br}}{U_\text{S}}\right|^2 = \frac{1}{9} \frac{(1 -
    \Omega^2)^2}{(1 - \Omega^2)^2 + 9 \Omega^2},
\end{equation}
%
und man erkennt, dass die \name{Wien}-\name{Robinson}-Brücke aus einem
kontinuierlichen Frequenzspektrum die Frequenz~$\omega_0$ herausfiltert.

\begin{figure}
  \centering
  \includegraphics[height=5cm]{wien-robinson-bruecke}
  \caption{Das Schaltbild der \name{Wien}-\name{Robinson}-Brücke
    entnommen aus \textcite{v302}. Die Brückenspannung ist
    frequenzabhängig, d.\,h. die Brücke kann als Frequenzfilter dienen.}
  \label{fig:wien-robinson-bruecke}
\end{figure}

\FloatBarrier
