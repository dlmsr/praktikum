% This work is licensed under the Creative Commons
% Attribution-NonCommercial 3.0 Unported License. To view a copy of this
% license, visit http://creativecommons.org/licenses/by-nc/3.0/.

\section{Auswertung}
\subsection{Dichtebestimmung der verwendeten Kugeln}
Die Gewichtsmessung ergibt für die kleine Kugel eine Masse von $m_\text{kl}$ = \SI{4.45}{\gram} und für die große Kugel eine Masse von $m_\text{gr}$ = \SI{4.95}{\gram}.\\
Die einmalige Vermessung der großen Kugel mit einer Schieblehre ergibt einen Durchmesser von $D_{gr}$ = \SI{15.8}{\milli\metre}. Für die kleine Kugel wird eine Messreihe mit Winkeln durchgeführt. Die Ergebnisse sind in Tabelle \ref{tab:durchmesser_kugel_kl} zu finden.\\
%
\begin{table}[h]
  \centering
  \begin{tabular}{SS|SS}
    \toprule
    {Nr.}  & {D/}\si{\milli\metre} & {Nr.}  & {D/}\si{\milli\metre}\\
    \midrule
1&15.22 & 6 & 15.22\\
2&15.28 & 7 & 15.26\\
3&15.08 & 8 & 15.16\\
4&15.22 & 9 & 15.12\\
5&15.24 & 10 & 15.22\\
    \bottomrule
  \end{tabular}
  \caption{Messwerte für den Durchmesser der kleinen Kugel}
  \label{tab:durchmesser_kugel_kl}
\end{table}
%
Als Durchmesser der kleinen Kugel ergibt sich ein Wert von  $D_{kl}$ = \SI{15.20}{\milli\metre} $\pm$ \SI{0.19}{\milli\metre}.\\
Aus den gemessenen Größen lässt sich die Dichte der Kugeln dadurch bestimmen, dass das Gewicht der Kugeln durch deren Volumen geteilt wird. In Tabelle \ref{tab:dichten} sind die Ergebnisse der Dichtebestimmung der beiden Kugeln zu finden. Bei der kleinen Kugel kann aufgrund der durchgeführten Messreihe der absolute Fehler durch eine \name{Gauss}sche Fehlerfortpflanzung angegeben werden.
%
\begin{table}[b]
  \centering
  \begin{tabular}{SSS}
    \toprule
   {Kugel} & $\rho${ in }\si{\gram\per\centi\metre^3}  &  {abs. Fehler in }\si{\gram\per\centi\metre^3}\\
    \midrule
{klein}& 2.42& 3e-5  \\
{groß}&2.40 &  {-}  \\
    \bottomrule
  \end{tabular}
  \caption{Ermittelte Dichten der beiden Kugeln}
  \label{tab:dichten}
\end{table}
%
\subsection{Apparaturkonstante der großen Kugel}
%
Um die Apparaturkonstante K mithilfe von Formel \eqref{eq:empirie} bestimmen zu können, wird zunächst die Viskosität des destillierten Wassers bei Zimmertemperatur (\SI{20}{\celsius}) bestimmt. In Tabelle \ref{tab:zeiten_const} sind die gemessenen Fallzeiten der kleinen und großen Kugel einzusehen. Die Apparaturkonstante der kleinen Kugel besitzt laut \textcite{v107} einen Wert von $K_{kl}$ = \SI{0.07640}{\milli\pascal\centi\metre^3\per\gram}. Der Wert für die Dichte des destillierten Wassers beträgt laut \textcite{wissenschaft-technik-ethik} $\rho_w$ = \SI{0.99820}{\gram\per\centi\metre^3}
bei einer Temperatur von \SI{20}{\celsius}.\\
%
\begin{table}[]
  \centering
  \begin{tabular}{S|S|S|S}
    \toprule
   \multicolumn{2}{c|}{kleine Kugel} & \multicolumn{2}{c}{große Kugel}\\
    \midrule
{Zeit t/s}&{Zeit t/s}&{Zeit t/s}&{Zeit t/s}\\
\midrule
11.92&12.06&79.1&78.95\\
11.93&12.06&78.61&78.95\\
12.12&11.96&78.32&78.10\\
12.32&12.16&79.03&79.06\\
12.13&11.95&78.16&78.10\\
11.84&11.95&78.40&78.83\\
12.07&11.92&78.16&78.00\\
12.21&12.16&78.89&78.84\\
12.20&12.04&77.64&78.20\\
12.10&12.26&78.40&78.72\\
    \bottomrule
  \end{tabular}
  \caption{Gemessene Fallzeiten der Kugeln bei Zimmertemperatur}
  \label{tab:zeiten_const}
\end{table}
%
Durch Verwenden von Formel \eqref{eq:empirie} kann mithilfe der Messwerte für die Fallzeiten der kleinen Kugel die Viskosität des destillierten Wassers bei Zimmertemperatur bestimmt werden.
Als Ergebnis erhält man, dass
$\eta$(\SI{20}{\celsius}) = \SI{1.31}{\milli\pascal\second} $\pm$ \SI{0.003}{\milli\pascal\second}.\\
Der Fehler wird mithilfe einer \name{Gauss}schen Fehlerforpflanzung der verwendeten Formel errechnet. Die gemessenen Zeiten sind hierbei durch einen statistischen Fehler behaftet.\\
%
Aus den gewonnenen Daten wird nun die Apparaturkonstante $K_{gr}$ für die große Kugel bestimmt.
Dazu wird ebenfalls Formel \eqref{eq:empirie} verwendet. Für die Viskosität wird der zuvor bestimmte Wert für die Viskosität des Wassers bei Zimmertemperatur verwendet.\\
Die Rechnung ergibt einen Wert von
\begin{equation*}
K_{gr} = \SI{0.011}{\milli\pascal\centi\metre^3\per\gram} \pm \SI{0.004}{\milli\pascal\centi\metre^3\per\gram}.
\end{equation*}
Bei der Fehlerfortpflanzung sind diesmal sowohl die Fallzeit, als auch die Viskosität mit einem Fehler behaftet.
%
\subsection{Temperaturabhängigkeit der dynamischen Viskosität}
%
Die gemessenen Fallzeiten der großen Kugel bei 10 verschiedenen Temperaturen ist in Tabelle \ref{tab:zeiten_var} angegeben.
%
\begin{table}[]
  \centering
  \begin{tabular}{SSSSS}
    \toprule
{Temperatur T /}\si{\celsius}&{Zeit t/s}&{Zeit t/s}&{Zeit t/s}&{Zeit t/s}\\
\midrule
25	&71.66&71.01&71.27&71.18\\
32	&65.12&63.00&64.45&64.47\\
35	&58.75&58.10&58.70&58.80\\
40	&54.04&54.13&54.33&52.61\\
45	&49.72&49.58&49.83&49.83\\
50	&46.00&45.55&45.69&45.78\\
55	&42.15&42.03&41.96&42.49\\
60	&39.55&39.69&39.52&39.38\\
65	&37.66&38.33&37.18&37.60\\
70	&36.10&36.24&35.44&35.94\\
    \bottomrule
  \end{tabular}
  \caption{Gemessene Fallzeiten der Kugeln bei verschiedenen Temperaturen}
  \label{tab:zeiten_var}
\end{table}
%
Mit Formel \eqref{eq:empirie} wird die Viskosität des Wassers bei den verschiedenen Temperaturen bestimmt. Es wird in der Formel die oben bestimmte Apparaturkonstante für die große Kugel verwendet, sodass bei der Fehlerfortpflanzung sowohl die Apparaturkonstante, als auch die Fallzeit fehlerbehaftet ist. Pro Temperatur wird der statistische Fehler der Fallzeit aus den vier Messwerten bei dieser Temperatur bestimmt.\\
In Tabelle \ref{tab:viskose} sind die so errechneten Viskositäten des verwendeten Wassers und die Fehler nachzulesen.\\
Ein graphischer Plot der Ergebnisse ist in Abb. \ref{fig:viskoseplot} zu sehen. Dabei wird $\log{(\eta)}$ gegen $\frac{1}{T}$ aufgetragen. Im selben Plot ist ebenfalls die Ausgleichsgerade durch die Messwerte zu finden.\\
Mit einer nicht-linearen Ausgleichsrechnung\footnote{Dazu wurde \texttt{ipython} in der Version 0.13  verwendet} werden die Koeffizienten in der \name{Andrade}schen Gleichung \eqref{eq:andrade} bestimmt. Die Ausgleichsrechnung ergibt, dass
\begin{equation*}
A = \SI{0.42}{\milli\pascal\second} \pm \SI{0.01}{\milli\pascal\second}
\end{equation*}
\begin{equation*}
B = \SI{28.03}{\celsius} \pm \SI{1.00}{\celsius},
\end{equation*}
sodass sich die gefundene Temperaturabhängigkeit der Viskosität des untersuchten Wassers beschreiben lässt durch
\begin{equation*}
\eta(T) = 0.42 \cdot \exp{\left(\frac{28.03}{T}\right)} \text{ }\si{\milli\pascal\second}.
\end{equation*}
%
Zuletzt wird noch die \name{Reynold}sche Zahl berechnet. Diese ist festgelegt als
\begin{equation*}
Re = \frac{v_m \cdot D \cdot \rho_{w}}{\eta},
\end{equation*}
mit $v_m$ $\hat{=}$ mittlere Geschwindigkeit der Kugel, D $\hat{=}$ Durchmesser der Kugel und $\rho_{w}$ $\hat{=}$ Dichte der durchströmten Flüssigkeit.\\
Bei Zimmertemperatur ergibt sich als Reynoldszahl für die kleine Kugel ein Wert von $\text{Re}_{kl}$ = \SI{96} und für die große Kugel ergibt sich, dass  $\text{Re}_{gr}$ = \SI{29}.
%
\begin{table}[]
  \centering
  \begin{tabular}{SSS}
    \toprule
{Temperatur T /}\si{\celsius}&{Viskosität /}\si{\milli\pascal\second}&{abs. Fehler /}\si{\milli\pascal\second}\\
\midrule
25	&1.189&0.002\\
32	&1.072&0.006\\
35	&0.978&0.003\\
40	&0.897&0.006\\
45	&0.830&0.001\\
50	&0.763&0.002\\
55	&0.703&0.002\\
60	&0.660&0.001\\
65	&0.629&0.004\\
70	&0.600&0.003\\
    \bottomrule
  \end{tabular}
  \caption{Errechnete Viskositäten für die jeweiligen Temperaturen}
  \label{tab:viskose}
\end{table}
%
\begin{figure}
\centering
\includegraphics{viskoseplot.pdf}
\caption{Plot der Messergebnisse $\log{(\eta)}$ gegen $\frac{1}{T}$}
\label{fig:viskoseplot}
\end{figure}
%