% This work is licensed under the Creative Commons
% Attribution-NonCommercial 3.0 Unported License.  To view a copy of
% this license, visit http://creativecommons.org/licenses/by-nc/3.0/.

\section{Theorie}

Eine Spannungsquelle ist ein elektrisches Gerät, das eine über die Zeit
konstante Leistung liefert.  Beispiele sind der Dynamo am Fahrrad, eine
Batterie oder ein Kraftwerk.  Kenngrößen einer Spannungsquelle sind
hierbei die Leerlaufspannung~$U_0$, die anliegt, wenn die
Spannungsquelle nicht durch einen Verbraucher belastet wird, und der
Innenwiderstand, der eingeführt werden muß, um das Verhalten der
Spannungsquelle bei Belastung, nämlich das Absinken der
Klemmenspannung~$U_\text{K}$ zu erklären.  Um eine Beziehung zwischen
den Größen Belastungswiderstand/-strom, Innenwiderstand,
Leerlaufspannung und Klemmenspannung herstellen zu können, wird das
Induktionsgesetz
\begin{equation}
  \label{eq:inductionsgesetz}
  \oint_{\partial A} \vec{E} \cdot \d\vec{s} = \iint_A \frac{\partial
    \vec{B}}{\partial t} \cdot \d\vec{A}
\end{equation}
aus den \name{Maxwell}-Gleichungen benötigt.  Bei Abwesenheit von
magnetischen Feldänderungen kann dieses Gesetz in eine einfachere Form
\begin{equation}
  \label{eq:maschenregel}
  \sum_j U_0^{(j)} = \sum_k R_k I_k,
\end{equation}
welche als sogenannte Maschenregel von \name{Kirchhoff} bezeichnet wird.

\begin{figure}[h]
  \centering
  \includegraphics{spannungsquelle}
  \caption{Hier ist das Ersatzschaltbild einer realen Spannungsquelle
    mit einem Belastungswiderstand in Reihe geschaltet.  Der
    gestrichelte Kasten kennzeichnet das Ersatzschaltbild einer realen
    Spannungsquelle. \cite{v301}}
  \label{fig:ersatzschaltung}
\end{figure}

In \cref{fig:ersatzschaltung} ist ein Ersatzschaltbild einer realen
Spannungsquelle zu sehen.  Die Maschenregel ergibt in diesem
Spezialfall:
\begin{align}
  U_0 = R_\text{i} I + R_\text{a} I\\
\intertext{oder}
  U_\text{K} = R_\text{a}I = U_0 - R_\text{i} I. \label{eq:klemme}
\end{align}
Hieraus ist unmittelbar ersichtlich, warum die Klemmenspannung mit
zunehmendem Belastungsstrom sinkt.  Aus Gleichung~\eqref{eq:klemme}
ergibt sich mit $I = U_\text{K}/R_\text{a}$:
\begin{equation}
  U_\text{K} = \frac{U_0}{1 - \frac{R_\text{i}}{R_\text{a}}}.
\end{equation}
Zur Messung der Leerlaufspannung wird ein Voltmeter mit einem im
Vergleich zum Innenwiderstand der Spannungsquelle großen Meßwiderstand
verwendet, wodurch für die Klemmenspannung im Grenzfall gilt:
\begin{equation}
  \lim_{R_\text{a}\to\infty} U_\text{K} = U_0.
\end{equation}
Die gemessene Klemmenspannung ist also eine Näherung für die
Leerlaufspannung. 

Aus der Existenz eines Innenwiderstandes ergibt sich auch die
Notwendigkeit einer sogenannten Leistungsanpassung.  Der Spannungsquelle
kann nämlich nur eine beschränkte Leistung entnommen werden, welche
abhängig vom gewählten Belastungswiderstand~$R_a$ ist.  Die Leistung die
am Verbraucher abgenommen wird, läßt als Funktion des
Belastungswiderstandes auf"-fassen:
\begin{equation}
  P = I^2 R_\text{a} = U_0^2 \frac{R_a}{(R_\text{i} + R_\text{a})^2}.
\end{equation}
An lokalen Extremstellen der Funktion ergibt sich notwendigerweise:
\begin{equation}
  \td{P}{R_\text{a}} = U_0^2 \frac{R_\text{i} -
    R_\text{a}}{(R_\text{i} + R_\text{a})^3} = 0 \iff R_\text{a} = R_\text{i}.
\end{equation}
Da überall $P \ge 0$ ist sowie $P(0) = 0$ und $P \to 0$ für $R_\text{a}
\to \infty$ gilt, erreicht die Leistung für $R_\text{a} = R_\text{i}$
ihr Maximum.
