% This work is licensed under the Creative Commons
% Attribution-NonCommercial 3.0 Unported License. To view a copy of this
% license, visit http://creativecommons.org/licenses/by-nc/3.0/.

\section{Durchführung}

\subsection{Bestimmung der Zeitkonstanten des RC-Gliedes}
Durch die Beobachtung des Auf- oder Entladevorgangs am Kondensator kann
die Zeitkonstante $RC$ bestimmt werden. Dazu wird ein Oszilloskop
verwendet, auf dem die Kondensatorspannung $U(t)$ gegen die Zeit
betrachtet wird. Es wird eine niedrige Frequenz eingestellt, so daß der
Kondensator genügend Zeit hat, sich vollständig zu laden und wieder zu
entladen. Hier kann nun an der Asymptotik der Kurve die Nullinie
abgelesen werden, die markiert wird.

Jetzt muß die Frequenz höher eingestellt werden, um Ausschnitte aus der
unendlich langen Entladekurve des Kondensators zu erhalten. Davon wird
ein Ausdruck angefertigt, aus dem dann Daten zur Bestimmung der
Zeitkonstanten entnommen werden können.

\subsection{Bestimmung der Amplitude der Kondensatorspannung in
  Abhängigkeit der Frequenz}

Zur Bestimmung der Abhängigkeit der Amplitude von der Frequenz wird ein
Millivoltmeter verwendet, daß die Spannung am Kondensator mißt. Über
ein geeignetes Spannungsgerät kann die Frequenz eingestellt und auch
abgelesen werden. Jetzt wird einfach eine Meßreihe Amplitude gegen
Frequenz durchgeführt.

\subsection{Bestimmung der Phasenverschiebung zwischen Generator- und
  Kondensatorspannung}

Um die Phasendifferenz $\varphi$ in Abhängigkeit der Frequenz $\omega$
zu erhalten, werden beide Signale am Oszilloskop betrachtet und mit dem
Cursor die zeitlichen Abstände $\Delta t$ zweier Nulldurchgänge der
beiden Schwingungen vermessen. Aus diesen zeitlichen Abständen erhält
man mit der Formel
%
\begin{equation}
  \label{eq:phase-zeitabstand}
  \varphi = 2\pi\frac{\Delta t}{T} \text{,}
\end{equation}
%
wobei $T$ die Periodendauer der Schwingung bezeichnet. Hierbei muß
darauf geachtet werden, daß beide Sinuskurven symmetrisch zur $x$-Achse
liegen.

\subsection{Untersuchung der Integrierfähigkeiten eines RC-Gliedes}

Wird die Frequenz der Spannungsquelle weiter hochgeregelt, kann
beobachtet werden, daß das RC-Glied als Integrator funktioniert. Es
werden hier mit dem Oszilloskop die Spannungen über dem Kondensator und
der Spannungsquelle betrachtet. Jetzt werden verschiedene Spannungen
eingestellt unter anderem Rechteckspannungen, Dreieckspannungen,
Sinusspannungen. Von diesen Signalen und ihren integrierten Signalen am
Kondensator werden Aufnahmen gemacht.
