% This work is licensed under the Creative Commons
% Attribution-NonCommercial 3.0 Unported License. To view a copy of this
% license, visit http://creativecommons.org/licenses/by-nc/3.0/.

\section{Auswertung}

\subsection{Bestimmung der mittleren Reichweite und des Energieverlustes}
\label{sec:mittlere-reichweite}

Die Messungen werden für die Abstände \SI{3}{\centi\metre} und
\SI{1.5}{\centi\metre} der Strahlenquelle zum Detektor durchgeführt.
Die aufgenommenen Werte finden sich in den
Tabellen~\ref{tab:impulse-3cm} und \ref{tab:impulse-1.5cm}. Eine
graphische Darstellung der Anzahl der Impulse in Funktion der effektiven
Weglänge ist in den Abbildungen~\ref{fig:impulse-3cm} und
\ref{fig:impulse-1.5cm} zu finden.

\begin{figure}
  \centering
  \includegraphics[width=0.8\textwidth]{abstand-3cm.pdf}
  \caption{Die Anzahl der Impulse in Funktion der effektiven Weglänge
    für den Abstand \SI{3}{\centi\metre}}
  \label{fig:impulse-3cm}
\end{figure}

\begin{figure}
  \centering
  \includegraphics[width=0.8\textwidth]{abstand-1_5cm.pdf}
  \caption{Die Anzahl der Impulse in Funktion der effektiven Weglänge
    für den Abstand \SI{1.5}{\centi\metre}}
  \label{fig:impulse-1.5cm}
\end{figure}

Aus der Messung im Abstand \SI{3}{\centi\metre} ergibt sich eine
mittlere Reichweite
%
\begin{equation}
  \label{eq:mittlere-reichw-3cm}
  R_m = \SI{23.7}{\milli\metre}\text{.}
\end{equation}

\begin{figure}
  \centering
  \includegraphics[width=0.8\textwidth]{abstand-3cm-energ.pdf}
  \caption{Energie in Funktion der effektiven Weglänge für den Abstand
    \SI{3}{\centi\metre}}
  \label{fig:energie-3cm}
\end{figure}

\begin{figure}
  \centering
  \includegraphics[width=0.8\textwidth]{abstand-1_5cm-energ.pdf}
  \caption{Energie in Funktion der effektiven Weglänge für den Abstand
    \SI{1.5}{\centi\metre}}
  \label{fig:energie-1.5cm}
\end{figure}

In Abbildung~\ref{fig:energie-3cm} findet sich die Energie der
Alpha-Teilchen in Funktion der effektiven Weglänge aufgetragen. Die rot
markierten Meßpunkte werden in eine lineare Regression\footnote{Für die
 lineare Regression wurde Scipy Version 0.11 verwendet.}  gegeben, da
sie auf einer Gerade zu liegen scheinen. Dieses Ergebnis ist ebenfalls
in der Abbildung zu sehen. Als Energieverlust der Alpha-Teilchen
berechnet sich damit zu:
%
\begin{equation}
  \label{eq:energieverlust}
  -\frac{\mathrm{d}E_\alpha}{\mathrm{d}x} = 
  \SI{0.124(5)}{\mega\electronvolt\per\milli\metre}
\end{equation}

Nachdem der Abstand auf \SI{1.5}{\centi\metre} verringert worden ist,
ergibt sich ein anderes Ergebnis. Bei \SI{0}{\milli\bar} liegt die
Anzahl der Impulse bei 154939. Bei Atmosphärendruck liegt die gemessene
Anzahl noch bei 136110, fällt also nicht auf die Hälfte der
Anfangsimpulse zurück. Wir kommen also nie in den Bereich, an dem
nurnoch die Hälfte alles $\alpha$-Teilchen vorhanden sind, können also
bei dieser Messreihe auch keinen Wert der mittleren Reichweite von
$\alpha$-Strahlung bestimmen.  Allerdings kann man mit Bestimmtheit
sagen, dass die mittlere Reichweite der $\alpha$-Strahlung in Luft
größer als \SI{1.5}{\centi\metre} ist.

Im Diagramm \ref{fig:energie-1.5cm} ist die Energie des Energiemaximums
gegen die effektive Länge aufgetragen. Die Energie lässt sich berechnen,
indem man von einer linearen Einteilung der Kanäle bezüglich der
gemessenen Stromimpulse ausgeht. Die Messwerte im Diagramm scheinen auf
einer Geraden zu liegen. Aus diesem Grund wird eine Regressionsgerade
durch alle Messpunkte gelegt. Die Steigung dieser Geraden entspricht
dem gesuchten Energieverlust $\mathrm{d}E_\alpha/\mathrm{d}x$ der
$\alpha$-Teilchen. Es ergibt sich ein Wert von
%
\begin{equation}
  \label{eq:wert-energie-verlust-1.5cm}
  -\frac{\mathrm{d}E_\alpha}{\mathrm{d}x} = 
  \SI{0.120(2)}{\mega\electronvolt\per\milli\metre}
\end{equation}

\begin{table}
  \centering\small
  \begin{tabular}{SSSSSS}
    \toprule
    {$p/\si{\milli\bar}$} & {Anz. d. Impulse} & {Kanal-Nr.} &
    {$p/\si{\milli\bar}$} & {Anz. d. Impulse} & {Kanal-Nr.} \\
    \midrule
    0 &  154939 & 2375 & 550 & 147817 & 1735\\
    50 &  154434 & 2272 & 600 & 146195 & 1687\\
    100 &  153985 & 2214 & 650 & 146318 & 1664\\
    150 &  152464 & 2215 & 700 & 144573 & 1536\\
    200 &  152032 & 2112 & 750 & 143582 & 1536\\
    250 &  151135 & 2055 & 800 & 143281 & 1455\\
    300 &  151902 & 2007 & 850 & 141832 & 1463\\
    350 &  151823 & 1979 & 900 & 140356 & 1366\\
    400 &  150482 & 1863 & 950 & 139431 & 1327\\
    450 &  150141 & 1867 & 1000 & 136110 & 1216\\
    500 &  148368 & 1792\\
    \bottomrule
  \end{tabular}
  \caption{Messwerte für den  Abstand \SI{1.5}{\centi\metre}}
  \label{tab:impulse-1.5cm}
\end{table}

\begin{table}
  \centering\small
  \begin{tabular}{SSSSSS}
    \toprule
    {$p/\si{\milli\bar}$} & {Anz. d. Impulse} & {Kanal-Nr.} &
    {$p/\si{\milli\bar}$} & {Anz. d. Impulse} & {Kanal-Nr.} \\
    \midrule
    0 & 56470 & 2387 &  550 & 51193 & 1295 \\
    50 & 56043 & 2351 &  600 & 50007 & 1135 \\
    100 & 55866 & 2215 &  650 & 49086 &  935 \\
    150 & 55289 & 1991 &  700 & 46759 &  847 \\
    200 & 55408 & 1943 &  750 & 39518 &  543 \\
    250 & 54734 & 1910 &  800 & 24424 &  323 \\
    300 & 54163 & 1792 &  850 &  7054 &  311 \\
    350 & 54241 & 1704 &  900 &   323 &  304 \\
    400 & 53700 & 1587 &  950 &    59 &  306 \\
    450 & 52655 & 1536 & 1000 &    29 &  312 \\
    500 & 52026 & 1455 &      &       &      \\
    \bottomrule
  \end{tabular}

  \caption{Messwerte für den Abstand \SI{3}{\centi\metre}}
  \label{tab:impulse-3cm}
\end{table}

\subsection{Statistische Auswertung des Alpha-Zerfalls}

\begin{figure}
  \centering
  \includegraphics[width=0.8\textwidth]{statistik.pdf}
  \caption{Statistische Messung zum Alpha-Zerfall}
  \label{fig:statistik}
\end{figure}

Abbildung~\ref{fig:statistik} stellt ein Histogramm mit den 100 erfaßten
Meßwerten dar. Diese Meßwerte finden sich auch in der
Tabelle~\ref{tab:statistik} wieder. Es soll nun bestimmt werden, ob
dieses Ergebnis sich besser durch eine \name{Poisson}- oder eine
\name{Gauss}-Verteilung beschreiben läßt.

Dazu schaut man sich den Mittelwert und die Standardabweichung unserer
Stichprobe an. $I$ bezeichnet hier die Anzahl der Impulse, die in
\SI{10}{\second} gemessen wurde.
%
\begin{align}
  \label{eq:statistik}
  \langle I\rangle &= \num{4556} & \sigma_I &= \num{157}
\end{align}
%
Nun kann man zu diesen errechneten Werten z.\,B. eine Normalverteilung
berechnen und mit den erhaltenen Daten vergleichen. Dabei verwendet man
den Mittelwert $\langle I\rangle$ als Schätzung für den Erwartungswert
und $\sigma_I$ also Schätzung für die Varianz. Mit
Formel~\eqref{eq:gauss}, die aus \textcite[823]{bronstein} entnommen
ist, ergibt sich dann die Kurve in Abbildung~\ref{fig:statistik}. Man
kann hier erkennen, daß eine Normalverteilung eine gute Beschreibung des
statistischen Vorgangs liefert.
%
\begin{equation}
  \label{eq:gauss}
  f(t) = \frac{1}{\sigma\sqrt{2\pi}}e^{-\frac{(t-\mu)^2}{2\sigma^2}}
\end{equation}
%

Im Gegensatz dazu sieht man, daß die
\name{Poisson}-Verteilung\footnote{Die Verteilung wurde mit Numpy
  Version 1.6.2 generiert.} deutlich weniger streut. Die
\name{Gauß}-Verteilung liefert hier eine bessere Approximation unserer
Stichprobe.

\begin{table}
  \centering\small
  \begin{tabular}{SSSSSSSSS}
    \toprule
    4643 & 4361 & 4682 & 4669 & 4394 & 4610 & 4422 \\
    4788 & 4658 & 4593 & 4438 & 4307 & 4523 & 4588 \\
    4491 & 4466 & 4715 & 4455 & 4648 & 4584 & 4823 \\
    4717 & 4582 & 4652 & 4304 & 4707 & 4377 & 4409 \\
    4721 & 4153 & 4467 & 4503 & 4449 & 4542 & 4783 \\
    4580 & 4635 & 4401 & 4479 & 4488 & 4549 & 4762 \\
    4653 & 4809 & 4372 & 4817 & 4732 & 4690 & 4466 \\
    4796 & 4830 & 4634 & 4374 & 4631 & 4781 & 4587 \\
    4707 & 4449 & 4405 & 4795 & 4520 & 4461 & 4288 \\
    4486 & 4830 & 4551 & 4340 & 4431 & 4534 & 4658 \\
    4607 & 4376 & 4587 & 4278 & 4536 & 4706 & 4530 \\
    4292 & 4719 & 4415 & 4498 & 4479 & 4595 & 4252 \\
    4647 & 4660 & 4657 & 4437 & 4490 & 4847 & 4557 \\
    4847 & 4557 & 4641 & 4500 & 4473 & 4628 & 4474 \\
    4294 & 4496 & 4266 & 4640 \\
    \bottomrule
  \end{tabular}
  \caption{Meßwerte zur Statistik des Alpha-Zerfalls}
  \label{tab:statistik}
\end{table}
