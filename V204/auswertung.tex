% This work is licensed under the Creative Commons
% Attribution-NonCommercial 3.0 Unported License. To view a copy of this
% license, visit http://creativecommons.org/licenses/by-nc/3.0/.

\section{Auswertung}
\subsection{Statische Messung}

In Abb. \ref{fig:t1undt4} ist ein Plot der gemessenen Temperaturen der
Thermoelemente T1 und T4 gegen die Zeit zu finden. In
Abb. \ref{fig:t5undt8} ist selbiges für die gemessenen Temperaturen der
Thermoelemente T5 und T8 abgebildet.  Aus der Betrachtung dieser Plots
stellt man fest, dass Die Temperaturen bei T1 und T4 zunächst ungefähr
gleich ansteigen, die Temperatur bei T1 allerdings immer etwas höher ist
als bei T4. Im späteren Verlauf scheinen die Kurven parallel zu
verlaufen.  Die Temperaturverläufe bei T5 und T8 hingegen unterscheiden
sich deutlich stärker voneinander. Die Temperatur bei T5 nimmt am Anfang
der Messung deutlich zu, wohingegen die Temperatur bei T8 kaum
ansteigt. Dennoch verlaufen auch hier die Kurven im späteren Verlauf
beinahe parallel zueinander, aber mit einem größeren
Temperaturunterschied als zwischen T1 und T4.

\begin{figure}
  \centering
  \includegraphics{T1undT4}
  \caption{Temperaturverläufe bei T1 und T4}
  \label{fig:t1undt4}
\end{figure}

\begin{figure}
  \centering
  \includegraphics{T5undT8}
  \caption{Temperaturverläufe bei T5 und T8}
  \label{fig:t5undt8}
\end{figure}
 
In Tabelle \ref{tab:700sekunden} sind die gemessenen Temperaturen der
äußeren Thermoelemente nach einer Messzeit von \SI{700}{\second} zu
finden. Es ist festzustellen, dass die Temperatur bei T5 nach
\SI{700}{\second} am größten ist, d.h., dass Aluminium von allen
verwendeten Metallen die größte Wärmeleitfähigkeit besitzt.

\begin{table}
  \centering
  \begin{tabular}{cS}
    \toprule
     T & {Temperatur in \si{\degreeCelsius}} \\
    \midrule
    1 &  42.73 \\
    4 &  40.67 \\
    5 &  45.26 \\
    8 &  32.93 \\
    \bottomrule
  \end{tabular}
  \caption{Temperaturen nach \SI{700}{\second}}
  \label{tab:700sekunden}
\end{table}

In Tabelle \ref{tab:waermestrom} sind für alle verwendeten Metallstäbe
die Wärmeströme $\frac{\Delta Q}{\Delta t}$ in \si{\joule\per\second}
nach jeweils \SI{200}{\second} angegeben, welche mit Formel
\eqref{eq:waermestrom} berechnet sind. Dabei wird für jedes Metall der
im folgenden Unterkapitel berechnete Wert für die Wärmeleitfähigkeit
$\kappa$ verwendet.

\begin{table}
  \centering
  \begin{tabular}{cSSSSS}
    \toprule
     Stab & {nach \SI{200}{\second}} & {nach \SI{400}{\second}} &
     {nach \SI{600}{\second}} & {nach \SI{800}{\second}} &
     {nach \SI{1000}{\second}}\\
    \midrule
    Messing (breit) & -0.193 & -0.625 & -0.504 & -0.485 & -0.483 \\
    Messing (schmal) & -0.109 & -0.349 & -0.301 & -0.295 & -0.295 \\
    Aluminium & -0.215 & -0.285 & -0.135 & -0.100 & -0.085 \\
    Edelstahl & -0.025 & -0.196 & -0.193 & -0.191 & -0.189 \\
    \bottomrule
  \end{tabular}
  \caption{Wärmestrom ${\Delta Q}/{\Delta t}$ in \si[per-mode=symbol]
    {\joule\per\second} der Stäbe}
  \label{tab:waermestrom}
\end{table}

Im Diagramm \ref{fig:t7-t8} ist die Differenz der Temperaturen der
Thermoelemente T7 und T8 zu sehen.  Die Kurve steigt zunächst stark an,
bis zu einem Wert von ca. \SI{8.3}{\degreeCelsius}. Dannach fällt die
Kurve leicht und scheint sich bei einem Wert von
ca. \SI{7.8}{\degreeCelsius} nicht mehr deutlich zu verändern. Das heißt
also, dass die Temperatur bei T7 zunächst ansteigt. Bei T8 setzt diese
Steigerung der Temperatur nach ca. \SI{180}{\second} ein. Da sich ein
annähernd konstanter Wert als Differenz der Temperaturen einstellt,
folgt daraus, dass zwischen T7 und T8 immer ungefähr eine Temperatur von
\SI{7.8}{\degreeCelsius} an die Umgebung abgegeben wird.

Im Diagramm \ref{fig:t2-t1} ist die Differenz der Temperaturen der
Theormoelemnte T2 und T1 abgebildet. Die Beobachtung bei dieser Kurve
weicht leicht von der Beobachtung der Kurve der Differenztemperatur
$\text{T7}-\text{T8}$ ab.  In Abb. \ref{fig:t2-t1} steigt die
Temperatur natürlich erst bei T2 an, bevor sie nach ca. \SI{60}{\second}
bei T1 ansteigt. Allerdings fällt die Differenztemperatur ab diesem
Zeitpunkt stark auf einen Wert von ca. \SI{2.5}{\degreeCelsius} ab. Es
werden hier also \SI{2.5}{\degreeCelsius} auf der Strecke zwischen den
beiden Thermoelementen abgegeben. Da die Umgebungstemperatur während der
Messreihe bei allen Stäben gleich war und der Abstand von T7 zu T8
gleich dem Abstand von T1 zu T2 ist, folgt daraus, dass Edelstahl eine
geringere Wärmeleitfähigkeit besitzt als Messing.


\begin{figure}
  \centering
  \includegraphics{T7-T8}
  \caption{Zeitlicher Verlauf der Differenztemperatur T7--T8}
  \label{fig:t7-t8}
\end{figure}

\begin{figure}
  \centering
  \includegraphics{T2-T1}
  \caption{Zeitlicher Verlauf der Differenztemperatur T2--T1}
  \label{fig:t2-t1}
\end{figure}

\subsection{Dynamische Messung}
Im Folgenden wird die Wärmeleitfähigkeit $\kappa$ für die verwendeten
Metallstäbe bestimmt. Dazu wird Formel \eqref{eq:kappa-aus-daempfung}
verwendet. Die Materialwerte, die beim Rechnen benutzt werden, findet
man in Tabelle~\ref{tab:materialkonst}.

\begin{table}
  \centering
  \begin{tabular}{lSSS}
    \toprule
     & {Messing} & {Aluminium} & {Edelstahl} \\
    \midrule
    $\rho\left[\si{\kilogram\per\cubic\metre}\right]$ & 8520 & 2800 &
    8000
    \\\addlinespace[1pt]
    $c\left[\si{\joule\per\kilogram\per\kelvin}\right]$ & 385 & 830 &
    400 \\\addlinespace[1pt]
    $\kappa\left[\si{\watt\per\metre\per\kelvin}\right]$ &
    118(14) & 313(28) & 15(1) \\\addlinespace[1pt]
    $A\left[\si{\milli\metre\squared}\right]$ (breit) & 48 & 48 & 48 
    \\\addlinespace[1pt]
    $A\left[\si{\milli\metre\squared}\right]$ (schmal)& 28 \\
    \bottomrule
  \end{tabular}
  \caption{Materialkonstanten der verwendeten Metalle}
  \label{tab:materialkonst}
\end{table}

Um die Phasendifferenzen zu bestimmen, wird immer die zeitliche
Verschiebung zweier aufeinanderfolgender Maxima bestimmt. Der Wert einer
Amplitude wird bestimmt, indem eine Gerade durch die beiden Minima
gelegt wird, die links und rechts von einem Maximum liegen. Anschließend
wird der Abstand zwischen der Gerade und dem Maximum
ausgerechnet. Dieses Vorgehen ist erforderlich, da der
\enquote{Nulldurchgang} der Schwingungen in der Zeit weder konstant noch
linear ist. Die Auswertung der Messwerte liefert die in
Tabelle~\ref{tab:materialkonst} dargestellten Ergebnisse für die
Wärmeleitfähigkeit $\kappa$.

\begin{figure}
  \centering
  \includegraphics{dyn_Messing}
  \caption{Zeitlicher Temperaturverlauf an T1 und T2}
  \label{fig:dyn_messing}
\end{figure}

\begin{figure}
  \centering
  \includegraphics{dyn_Aluminium}
  \caption{Zeitlicher Temperaturverlauf an T5 und T6}
  \label{fig:dyn_aluminium}
\end{figure}

\begin{figure}
  \centering
  \includegraphics{dyn_Edelstahl}
  \caption{Zeitlicher Temperaturverlauf an T7 und T8}
  \label{fig:dyn_edelstahl}
\end{figure}

In den Abbildungen~\vrefrange{fig:dyn_messing}{fig:dyn_edelstahl} sieht
man den zeitlichen Verlauf der Temperatur graphisch dargestellt.
