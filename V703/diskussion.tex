% This work is licensed under the Creative Commons
% Attribution-NonCommercial 3.0 Unported License.  To view a copy of
% this license, visit http://creativecommons.org/licenses/by-nc/3.0/.

\section{Diskussion}

Bei der Bestimmung der Charakteristik ist ein Fehler gemacht worden,
daher ist das Ergebnis in diesem Versuchsteil wenig aussagekräftig. Zwar
ist versucht worden, den systematischen Fehler herauszurechnen,
allerdings ist die genaue Verschiebung der Meßwerte durch die
Veränderung des Probenabstands natürlich nicht bekannt.

Bei der Bestimmung der Totzeit liegen die Ergebnisse beider Methoden in
derselben Größenordnung. Die Totzeit, die durch die Zwei-Quellen-Methode
bestimmt worden ist, liegt einen Faktor 5 über der Totzeit, die mit dem
Oszilloskop bestimmt worden ist. Der Wert, der durch die
Oszilloskop-Messung bestimmt wurde, ist aber realistischer
einzuschätzen.

Die Werte, die für die Ladungsfreisetzung pro Teilchen erhalten worden
sind, sind als unerwartet einzuschätzen.  Das verwendete Präparat war
ein Beta-Strahler, d.\,h. die Strahlungsteilchen sind Elektronen.  Da
die Zählrohrspannung bekannt ist, kann die Energie, die die Elektronen
zusätzlich aufgenommen haben, mithilfe dieser abgeschätzt werden.  Sie
müßte im Bereich einiger Hundert \si{\electronvolt} liegen.  Ionisiert
dieses Elektron die Argon-Atome, dessen Elektronen Bindungsenergien im
Bereich einiger \SI{10}{\electronvolt} liegen, so müßten bei der
bestimmten Ladungsmenge pro Teilchen ca. \num{5e10} Argon-Atome
ionisiert werden. Dies könnte damit erklärt werden, daß die
Bindungsenergie der Nukleonen, woraus das Elektron seine Energie
mitbekommt, deutlich größer als die Bindungsenergie im Atom ist.  Die
hohe Ladungsfreisetzung pro Teilchen könnte möglicherweise aber auch
durch Sekundärionisation oder durch UV-Photonen erklärt werden.
