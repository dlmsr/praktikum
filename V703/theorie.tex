% This work is licensed under the Creative Commons
% Attribution-NonCommercial 3.0 Unported License.  To view a copy of
% this license, visit http://creativecommons.org/licenses/by-nc/3.0/.

\section{Theorie}
\subsection{Funktionsweise eines Geiger-Müller-Zählrohres}
%
Das Geiger-Müller-Zählrohr wird zum Messen von Intensitäten ionisiernder
Stahlung verwendet. Es besteht aus einem zylinderförmigen Stahlmantel,
welcher z.B. mit \SI{100}{\milli\bar} Argon und \SI{10}{\milli\bar}
Ethylalkohol gefüllt ist, einem Eintrittsfenster und einem Draht,
welcher sich mittig im Zählrohr befindet. Das Eintrittsfenster des
Zählrohres in diesem Versuch besteht aus Mylar-Folie, da diese eine so
geringe Dichte besitzt, dass selbst niederreichweitige Strahlung durch
diese hindurchdringen kann.  Der Mantel wird als Kathode, und der innere
Draht als Anode verwendet, sodass beim Eintritt ionisierender Strahlung
die entstehenden Ionen zum Mantel und die Elektronen zum Draht aufgrund
der wirkenden elektrischen Kraft wandern, sodass bei einem geschlossenen
Stromkreis ein Impuls messbar ist.

In Abb.~\ref{fig:zaehlrohr_aufbau} ist ein Querschnitt eines Zählrohres
zu sehen.
%
\begin{figure}[b]
  \centering
  \includegraphics[width=0.7\textwidth]{zaehlrohr_aufbau}
  \caption{Querschnitt durch ein Geiger-Müller-Zählrohr.
               Die Skizze wurde aus \textcite{v703} entnommen.}
  \label{fig:zaehlrohr_aufbau}
\end{figure}
%

Die Anzahl der entstehenden Elektronen-Ionen-Paare hängt stark von der
angelegten Spannung zwischen Mantel und Innendraht ab.  In
Abb.~\ref{fig:ionenpaare} ist der Zusammenhang zwischen angelegter
Spannung und der Anzahl der Paare zu sehen.
%
\begin{figure}
  \centering
  \includegraphics[width=0.5\textwidth]{ionenpaare}
  \caption{Die Anzahl der erzeugten Elektron-Ionenpaare gegen die
    angelegte Spannung U zwischen Anodendraht und Kathodenmantel des
    Zählrohres. Der Geiger-Müller-Bereich ist der in diesem Versuch
    interessante Bereich.  Die Skizze wurde aus \textcite{v703}
    entnommen.}
  \label{fig:ionenpaare}
\end{figure}
\subsection{Charakteristik des Zählrohres}
%

Als eine der Charakteristiken eines Zählrohres wird der
Geiger-Müller-Bereich in Abb.~\ref{fig:ionenpaare} bezeichnet. Dort ist
die dargestellte Kurve eine Gerade, welche als Plateau bezeichnet
wird. Je geringer die Steigung dieses Plateaus ist, desto hochwertiger
ist das Zählrohr. Die nicht verschwindende Steigung kommt dadurch
Zustande, dass es im Zählrohr zu mehr Nachentladungen kommt, je höher
die Spannung zwischen Mantel und Innendraht eingestellt ist.
%
\subsection{Totzeit und Nachentladungen eines Zählrohres}
%

Tritt Strahlung in das Geiger-Müller-Zählrohr ein und ionisiert das
darin befindliche Gas, wandern die frei gewordenen Elektronen zum
Anodendraht, die positiven Ionen zum Mantel. Da letztere eine deutlich
größere Masse als die Elektronen besitzen, halten sie sich wesentlich
länger zwischen Anode und Kathode auf. Dadurch bildet sich ein
sogenannter Ionenschlauch, welcher das elektrische Feld abschirmt. Tritt
während dieser Zeit weitere Strahlung in das Zählrohr ein und ionisiert
die Gasmoleküle, so wird dies nicht registiert, da die Elektronen nicht
abgesaugt werden und nach einiger Zeit wieder mit den positiven Ionen
rekombinieren.

Diese Zeit, in der einfallende ionisierende Strahlung nicht registiert
werden kann, wird Totzeit $T$ genannt.

Nach dieser Totzeit kann Strahlung wieder registiert werden, allerdings
sind die Ladungsimpulse für eine weitere Zeit, die sogenannte
Erholungszeit $T_E$, schwächer.  Erst wenn alle positiven Ionen abgebaut
wurden, sind das elektrische Feld und die Ladungsimpulse wieder maximal.
In Abb.~\ref{fig:totzeit} sind die Ladungsimpulshöhe gegen die Zeit
aufgetragen. Es sind ebenfalls die Tot- und Erholungszeit eingezeichnet.
%
\begin{figure}
  \centering
  \includegraphics[width=0.7\textwidth]{totzeit}
  \caption{Diese Skizze zeigt das Bild, welches in etwa
    auf dem Oszilloskop angezeigt wird. Die Tot- und
    Erholungszeit können damit leicht bestimmt werden.
    Die Skizze wurde aus \textcite{v703} entnommen.}
  \label{fig:totzeit}
\end{figure}
%

Auch wenn während der Erholungszeit keine weiteren Teilchen in das
Zählrohr eintreten, kann es zu Ladungsimpulsen kommen. Die auf die
Mantelfläche des Zählrohres auftreffenden Ionen können bei ihrer
Neutralisation so viel Energie freisetzen, dass weitere Elektronen,
sogenannte Sekundärelektronen, freigesetzt werden, welche zum
Anodendraht hin beschleunigt werden und somit zu einer weiteren Zündung
des Gases führen können, welches als Ladungsimpuls registiert wird,
obwohl keine Strahlung in das Zählrohr eingetreten ist.

Um den Effekt der Nachentladung zu minimieren, wird Alkoholdampf
verwendet. Die Argonionen treffen auf ihrem Weg zum Mantel auf diese und
neutralisieren sich an ihnen. Nun werden die Alkoholionen zum Mantel hin
beschleunigt. Werden die Alkoholionen am Mantel neutralisiert, können
sie die freigewordene Energie in Schwingung umsetzen, sodass es deutlich
seltener zu Nachentladungen kommt.
%
\subsection{Das Ansprechvermögen}
%
Das Ansprechvermögen bezeichnet die Wahrscheinlichkeit dafür, dass in
das Zählrohr einfallende ionisierende Strahlung mit dem dortigen Gas
wechselwirkt und somit vom Zählrohr registriert werden kann.

Die $\beta$-Teilchen haben aufgrund ihres hohen Ionisationsvermögen ein
Ansprechvermögen von ca. \SI{100}{\percent}.  Deswegen werden die
Messungen in diesem Versuch nur mit $\beta$-Strahlern durchgeführt.

\FloatBarrier
%