% This work is licensed under the Creative Commons
% Attribution-NonCommercial 3.0 Unported License. To view a copy of this
% license, visit http://creativecommons.org/licenses/by-nc/3.0/.

\section{Auswertung}
%
\subsection{Elektronenstrahl im E-Feld}
%
Die Elektronen werden in diesem Versuch in die y-Achse der Kathodenstrahlröhre mithilfe von Ablenkplatten beschleunigt, weswegen die Abmessungen dieser verwendet werden. Der Abstand L zwischen den Ablenkplatten und dem Schirm beträgt \SI{0.143}{\metre}, die Plattenlänge p = \SI{0.019}{\metre} und der Plattenabstand d ist zu \SI{0.0038}{\metre} bemessen.
%
\subsubsection{Empfindlichkeit der Kathodenstrahlröhre}
%
Die Empfindlichkeit E einer Kathodenstrahlröhre bezeichnet den Quotienten $\frac{D}{U_d}$, also die Ablenkung pro angelegter Ablenkspannung. Tabelle \ref{tab:messwerte_e_1} beinhaltet die gemessenen Ablenkspannungen $U_\text{d}$ zu den verschiedenen Abständen y des Leuchtflecks von der x-Achse für die Beschleunigungsspannungen $U_B$ = \SI{200}{\volt}, \SI{260}{\volt}, \SI{320}{\volt}, \SI{440}{\volt} und \SI{500}{\volt}. In Abb. \ref{fig:empfindlichkeit} ist ein Plot der aufgenommenen Ablenkungen des Elektronenstrahls gegen die angelegte Ablenkspannung für die oben genannten Beschleunigungsspannungen abgebildet. Im selben Plot sind auch die jeweiligen Ausgleichsgeraden zu finden.
%
\begin{table}[h]
  \centering
  \begin{tabular}{S|S|S|S|S|S}
    \toprule
    $U_\text{B}${ in V}& $U_\text{d}${ in \si{\volt}} & $\Delta ${y in \si{\milli\metre}}& $U_\text{B}${ in V}& $U_\text{d}${ in \si{\volt}} & $\Delta ${y in \si{\milli\metre}}\\
    \midrule
200 & -10.13& 25.4 &         & 4.98& 6.35 \\ 
      &-6.21& 19.05 &          & 9.58& 0 \\
      & -2.04& 12.7 &           & 15.92& -6.35 \\ 
      & 2.76& 6.35 &            & 21.8& -12.7\\ 
      & 6.59& 0 &                & 28.1& -19.05\\  
      & 10.21& -6.35 &        & 33.4& -25.4\\  
      & 14.14& -12.7 &  440  & -21.4& 25.4\\  
      & 17.60& -19.05 &      & -10.97& 19.05\\  
      & 21.2& -25.4 &         & -1.2&12.7\\  
260 & -13.48&25.4 &        & 5.94& 6.35\\ 
      & -8.3& 19.05 &        & 14.6& 0\\  
      & -2.99& 12.7 &         & 21.3& -6.35\\ 
      & 3.09& 6.35 &         & 30.1&-12.7\\
      & 8.95& 0 &              & 35.6&-16.51\\ 
      & 14.11& -6.35 & 500  & -23.5& 25.4\\  
      & 18.65& -12.7 &        & -14.19& 19.05\\  
      & 23.4& -19.05 &        & -3.18& 12.7\\ 
      & 28.9& -25.4 &        & 6.89&6.35\\ 
320 & -15.7& 25.4 &         & 16.4&0\\  
      & -9.84& 19.05 &      & 25.1& -6.35\\
      & -2.75& 12.7 &          &34.7&-12.7\\
 \bottomrule
  \end{tabular}
  \caption{Messwerte zur Empflindlichkeit der Kathodenstrahlröhre}
  \label{tab:messwerte_e_1}
\end{table}
%
Die Koeffizienten und deren Fehler der Ausgleichsgerade werden durch eine lineare Ausgleichsrechnung bestimmt\footnote{Dazu wurde \texttt{ipython}
 in der Version 0.13  verwendet}. Im Nachfolgenden sind die Funktionen der Ausgleichsgeraden für die verschiedenen Beschleunigungsspannungen angegeben; dabei bezeichnet der Index der Funktionen die zugehörige Beschleunigungsspannung.
\begin{eqnarray*}
D_{200}(x) = \SI{-0.00161}{\metre\per\volt} \cdot x + \SI{0.0097}{\metre}\\
D_{260}(x) = \SI{-0.00119}{\metre\per\volt} \cdot x +\SI{ 0.0096}{\metre} \\
D_{320}(x) = \SI{-0.00103}{\metre\per\volt} \cdot x +\SI{ 0.0098}{\metre} \\
D_{440}(x) = \SI{-0.00076}{\metre\per\volt} \cdot x +\SI{ 0.0105 }{\metre} \\
D_{200}(x) = \SI{-0.00065}{\metre\per\volt} \cdot x + \SI{0.0103 }{\metre} \\
\end{eqnarray*}
%
In Tabelle \ref{tab:empfindlichkeiten} sind die errechneten Empfindlichkeiten E, sowie deren Fehler, zu den verschiedenen Beschleunigungsspannungen $U_\text{B}$ zu finden.

%
\begin{figure}[h]
\centering
\includegraphics{empfindlichkeit.pdf}
\caption{Ablenkung D gegen Ablenkspannung $U_d$}
\label{fig:empfindlichkeit}
\end{figure}
%
\begin{table}
  \centering
  \begin{tabular}{S S S}
    \toprule
    $U_B${ in V}& {|E| in \si{\milli\metre\per\volt}} & $\Delta ${E in \si{\milli\metre\per\volt}}\\
    \midrule
     200 & 1.61& 0.03 \\
     260 & 1.19& 0.02 \\
     320 & 1.03& 0.02 \\
     440 & 0.76& 0.02 \\
     500 & 0.65& 0.01 \\
 \bottomrule
  \end{tabular}
  \caption{Ermittelte Empfindlichkeiten der Kathodenstrahlröhre}
  \label{tab:empfindlichkeiten}
\end{table}
%
Ein Plot dieser ermittelten Werte, sowie eine Ausgleichsgerade durch diese ist in Abb. \ref{fig:a} zu finden.
%
\begin{figure}
\centering
\includegraphics{a.pdf}
\caption{Ermittelte Empfindlichkeiten E gegen $U_B^{-1}$}
\label{fig:a}
\end{figure}
%
Die Steigung a der Ausgleichsgeraden in Abb. \ref{fig:a} beträgt \SI{-0.3109}{\metre} $\pm$ \SI{0.0116}{\metre}. Steckt man die Anschlüsse des Multimeters um, so ergibt sich ein positiver Wert für a. Aufgrund von theoretischen Überlegungen sollte dieser Wert mit der Größe $\frac{pL}{2d}$ übereinstimmen. Es ergibt sich, dass  $\frac{pL}{2d}$ = \SI{0.3575}{\metre} ist. Die Geradengleichung der Ausgleichsgeraden in Abb. \ref{fig:a} lautet
\begin{equation*}
E(x) = \SI{-0.31}{\metre} \cdot \SI{-3.7e-5}{\metre\per\volt}.
\end{equation*}
%
\subsubsection{Kathodenstrahl-Oszillograph}
%
Nahezu unbewegte Bilder auf dem Leuchtschirm erhält man bei den in Tabelle \ref{tab:freq} angegebenen Werten. Dort sind ebenfalls die dazugehörigen Amplituden des Sinusbildes zu finden. Als Frequenz $\nu$ des Sinusgenerators errechnet sich als Mittelwert mit Standardabweichung 
\begin{equation*}
\nu  = \SI{79.7}{\hertz} \pm \SI{0.1}{\hertz}.
\end{equation*}
%
\begin{table}[h]
  \centering
  \begin{tabular}{S S S}
    \toprule
    {Frequenzverhältnis n}& $\nu${ in \si{\hertz}} & $A_{\text{FPP}}${ in \si{\milli\metre}}\\
    \midrule
     0.5 &39.9& 31.75 \\
     1 & 79.5&31.75 \\
     2 & 159.5& 29.21 \\
     3 & 239.2& 29.21 \\
 \bottomrule
  \end{tabular}
  \caption{Gemessene Werte bei stehendem Bild}
  \label{tab:freq}
\end{table}
%

Bei der Verwendung des im vorherigen Abschnitts Wertes für a, ergibt sich als Scheitelwert $\hat{U}$ des Sinusgenerators zu 
\begin{equation*}
\hat{U} = \SI{3.86}{\volt} \pm\SI{0.22}{\volt}.
%
\end{equation*}
Der Fehler hierbei wird mithilfe der in Formel \eqref{eq:gauss_e} angegebenen Gau\ss schen Fehlerforpflanzung berechnet.

\begin{equation}
\label{eq:gauss_e}
\Delta \hat{U} = \sqrt{\left(\frac{-\bar{D} \cdot U_B}{\bar{a}^2} \cdot \Delta a\right)^2 + \left(\frac{U_B}{\bar{a}}  \cdot \Delta D\right)^2}
\end{equation}
%
\subsection{Elektronenstrahl im B-Feld}
%
Die in diesem Versuch verwendete Kathodenstrahlröhre gleicht in ihrem Aufbau und Abmessungen der Röhre, die im vorherigen Versuch verwendet wurde.
%
\subsubsection{Spezifische Ladung des Elektrons}
%
In Abb. \ref{fig:b} sind der Quotient $\frac{D}{L^2 + D^2}$ gegen die verwendeten Magnetfeldstärken B bei zwei verschiedenen Beschleunigungsspannungen, sowie eine Ausgleichsgerade durch die Messwerte aufgetragen. Die dazugehörigen Messwerte sind in Tabelle \ref{tab:messwerte_b_a} zu finden. Ebenfalls in dieser Tabelle zu finden sind die Magnetfeldstärken zu den verwendeten Stromstärken und der Quotient $\frac{D}{L^2 + D^2}$. Die Stärke des Magnetfeldes eines Helmholtzspulenpaares mit N Windungen und dem Radius R erhält man aus der verwendeten Stromstärke I nach Formel \eqref{eq:mag-force-helmholtz}. Aus dem Proportionalitätsfaktor A ergibt sich nach Formel \eqref{eq:spec-charge-measurement} die spezifische Ladung $\frac{e_0}{m_0}$ durch die in Gleichung \eqref{eq:spez_Ladung} wiedergegebene Rechenvorschrift. Die Funktion der Ausgleichsgeraden in Abb. \ref{fig:b} ergibt sich durch eine lineare Ausgleichsrechnung zu 
%
\begin{equation*}
f(x) = \SI{13030.4}{\per\metre\tesla} \cdot x + \SI{0.11}{\per\metre}
\end{equation*}
für die Beschleunigungsspannung von \SI{250}{\volt} und zu
%
\begin{equation*}
f(x) = \SI{9872.4}{\per\metre\tesla} \cdot x +  \SI{0.04}{\per\metre}.
\end{equation*}
%
für \SI{450}{\volt} Beschleunigungsspannung.
%
\begin{table}
  \centering
  \begin{tabular}{S|S|S|S|S}
    \toprule
    $U_\text{B}${ in V}&{D in \si{\milli\metre}} &{I in \si{\ampere}}& {B in \si{\micro\tesla}} & $\frac{D}{D^2 + L^2}$ { in \si{\per\metre}} \\
    \midrule
250 &0     & 0     &   0 & 0\\
      & 6.35&0.24 & 15.3& 0.31\\
      & 12.7& 0.51 & 32.5&0.62\\
      & 19.05& 0.99 & 63.1&0.92\\
      &25.4& 1.26 & 80.4&1.20\\
      & 31.75&1.6 & 102.0&1.48\\
      & 38.1& 1.9 & 121.2&1.74\\
      & 44.45& 2.3 & 146.7&1.98\\
      & 50.8& 2.6 & 165.8&2.21\\
450 &0     & 0     & 0      &0\\
      & 6.35&0.45 & 28.7   &0.31\\
      & 12.7& 0.9 & 57.4& 0.62\\
      & 19.05& 1.35 & 86.1&0.92\\
      &25.4& 1.8 & 114.8&1.20\\
      & 31.75&2.25 & 143.5&1.48\\
      & 38.1& 2.7 & 172.2&1.74\\
      & 44.45& 3.1 & 197.7&1.98\\
      & 46.0375& 3.25 & 207.3&2.04\\
 \bottomrule
  \end{tabular}
  \caption{Messwerte zur Bestimmung der spezifischen Ladung}
  \label{tab:messwerte_b_a}
\end{table}
%
\begin{equation}
\label{eq:spez_Ladung}
\frac{e_0}{m_0} = 8 \cdot U_B \cdot A^2
\end{equation}
%
\begin{figure}
\centering
\includegraphics{b.pdf}
\caption{Zur spezifischen Ladung: Quotient $\frac{D}{L^2 + D^2}$ in 1/m gegen B in T}
\label{fig:b}
\end{figure}
%
Als Steigung dieser Geraden errechnen sich die in Tabelle \ref{tab:steigungen} wiedergegebenen Werte. Ebenfalls in dieser Tabelle zu finden sind die daraus resultierenden Werte für die spezifische Ladung des Elektrons, sowie die Fehler der einzelnen Ergebnisse. Der Fehler der spezifischen Ladung wird mithilfe einer Gau\ss schen Fehlerfortpflanzung von \eqref{eq:spez_Ladung}berechnet. Dabei ist A, also die Steigung der Geraden in Abb. \ref{fig:b} die Fehlerbehaftete Größe.
%
\begin{table}
  \centering
  \begin{tabular}{S S S S}
    \toprule
    $U_B${ in V} & {A in \si{\metre\per\volt\second}} & $\frac{e_0}{m_0}${ in \si{\coulomb\per\kilo\gram}} & $\Delta \frac{e_0}{m_0}${ in \si{\coulomb\per\kilo\gram} }\\
    \midrule
     250 &13030.4& 3.4 $\cdot 10^{11}$ & 2.1 $\cdot 10^{10}$ \\
     450 & 9872.4& 3.5 $\cdot 10^{11}$& 1.0 $\cdot 10^{10}$ \\
 \bottomrule
  \end{tabular}
  \caption{Proportionalitätsfaktoren und spez. Ladung}
  \label{tab:steigungen}
\end{table}
%
Somit ergibt sich in diesem Versuch der Wert der spezifischen Ladung zu
\begin{equation*}
\frac{e_0}{m_0} = \SI{3.45e11}{\coulomb\per\kilo\gram} \pm \SI{0.16e11}{\coulomb\per\kilo\gram}
\end{equation*}
\subsubsection{Lokale Erdmagnetfeldstärke}
%
Nach der Drehung des Kathodenstrahlrohres von Nord-Süd-Richtung in Ost-West-Richtung wird eine Stromstärke von  \SI{0.085}{\ampere} in den verwendeten Helmholtzspulen benötigt, um mit diesen ein Magnetfeld zu erzeugen, welches die durch das Erdmagnetfeld verursachte Ablenkung des Strahls ausgleicht. Nach Formel \eqref{eq:mag-force-helmholtz} ergibt sich somit für die horizontale Magnetfeldstärke $B_\text{hor}$ ein Wert von \SI{5.4}{\micro\tesla}.

Um nun die Gesamtfeldstärke zu berechnen wird der Inklinationswinkel $\phi$ des Magnetfeldes in Versuchsnähe benötigt. In Tabelle \ref{tab:winkel} sind die in einer Messreihe bestimmten Inklinationswinkel angegeben. Als Mittelwert ergibt sich $\phi$ $\approx$ \SI{70.5}{\degree} $\pm$ \SI{9.4}{\degree}. Daraus lässt sich leicht mithilfe von Trigonometrie die gesamte lokale Erdmagnetfeldstärke B bestimmen. Es ergibt sich, dass 
\begin{equation*}
B = \SI{16.2}{\micro\tesla}  \pm \SI{7.5}{\micro\tesla}.
\end{equation*}
%
\begin{table}[h]
  \centering
  \begin{tabular}{S S}
    \toprule
    {Messung Nr.}& {Inklinationswinkel in \si{\degree}}\\
    \midrule
     1 &70\\
     2 & 86 \\
     3 &40\\
     4 & 86 \\
 \bottomrule
  \end{tabular}
  \caption{Messwerte zur Bestimmung des Inklinationswinkels}
  \label{tab:winkel}
\end{table}
%