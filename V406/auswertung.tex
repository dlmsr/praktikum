% This work is licensed under the Creative Commons
% Attribution-NonCommercial 3.0 Unported License. To view a copy of this
% license, visit http://creativecommons.org/licenses/by-nc/3.0/.

\section{Auswertung}

\begin{figure}
  \centering
  \includegraphics{geometrie}
  \caption{Geometrie der Apparatur}
  \label{fig:apparatur-geometrie}
\end{figure}

\section{Einzelspalt}

Bei der Messung am Einzelspalt, kommt ein Einzelspalt mit einer Weite
von ca. \SI{0.08}{\milli\metre} zum Einsatz. Die gewonnenen Meßwerte
(diese sind angehängt) werden nun in eine nichtlineare Regression
gegeben\footnote{hierzu wird die Funktion
  \texttt{scipy.optimize.curve\_fit} aus der \texttt{scipy}-Bibliothek
  in der Version 0.11 verwendet}. Aus
Formel~\eqref{eq:intensität_einzelspalt}, die die Intensität beschreibt,
und der Abbildung~\ref{fig:apparatur-geometrie} ergibt sich dieser
Zusammenhang zwischen der gemessenen Stromstärke $I$ und der Strecke $x$
%
\begin{equation}
  \label{eq:strom-einzelspalt}
  I(x) \propto A_0\frac{L\lambda}{\pi(x-c)} \sin\left(\frac{\pi b}{\lambda
        L} (x-c)\right)
\end{equation}
%
Die nichtlineare Regression führt auf den Funktionsterm
%
\begin{equation}
  \label{eq:einzelspalt_fit}
  \sisetup { exponent-to-prefix = true }
  I(x) = \SI{8.828}{\ampere\per\metre}
  \frac{\lambda L}{\pi(x-\SI{27.2e-3}{\metre})}
  \sin\left(\frac{\pi\cdot\SI{0.0778e-3}{\metre}}{\lambda L} 
    (x-\SI{27.2e-3}{\metre})\right)
\end{equation}
%
mit den Werten $\lambda = \SI{633}{\nano\metre}$ und $L =
\SI{1}{\metre}$.  In Abbildung~\ref{fig:single-slit} sind die Meßwerte
und der Graph der erhaltenen Funktion dargestellt.

\begin{figure}
  \centering
  \includegraphics[width = 0.8\textwidth]{single-slit}
  \caption{Meßwerte und Graph des Fits beim Einzelspalt}
  \label{fig:single-slit}
\end{figure}

\section{Variabler Spalt}

Für den variablen Spalt wird die gleiche Funktion aus
Formel~\eqref{eq:strom-einzelspalt} mit den erhaltenen Meßwerten
(ebenfalls im Anhang) in eine nichtlineare Regression gegeben. Es
wird Funktionsterm
%
\begin{equation}
  \label{eq:variable-slit-fit}
  \sisetup { exponent-to-prefix = true }
  I(x) = \SI{4.189}{\ampere\per\metre}
  \frac{\lambda L}{\pi(x-\SI{26.2e-3}{\metre})}
  \sin\left(\frac{\pi\cdot\SI{0.0503e-3}{\metre}}{\lambda L} 
    (x-\SI{26.2e-3}{\metre})\right)
\end{equation}
%
mit den Werten $\lambda = \SI{633}{\nano\metre}$ und $L =
\SI{1}{\metre}$ erhalten. Die Strecke $L$ ist durch Messung bekannt. Die
Wellenlänge des He-Ne-Lasers ist in \textcite[36]{v406} angegeben. In
Abbildung~\ref{fig:variable-slit} sind die Meßwerte mit dem Graphen der
erhaltenen Funktion dargestellt.

\begin{figure}
  \centering
  \includegraphics[width = 0.8\textwidth]{variable-slit}
  \caption{Meßwerte und Graph des Fits beim variablen Spalt}
  \label{fig:variable-slit}
\end{figure}

\section{Doppelspalt}

Die Geometrie beim Doppelspalt ist die gleiche wie beim Einzelspalt in
Abbildung~\ref{fig:apparatur-geometrie}. Aus dieser kann man auch hier
erkennen, daß 
%
\begin{equation}
  \label{eq:x-phi}
  \sin\phi = \frac{x}{L}\text{.}
\end{equation}
%
Aus Formel~\eqref{eq:intensität_doppelspalt} und \eqref{eq:x-phi} ergibt
sich dann folgender Zusammenhang zwischen der gemessenen Strecke $x$ und
der Stromstärke $I(x)$:
%
\begin{equation}
  \label{eq:double-slit-current}
  I(x) \propto \left( \frac{2\lambda L}{\pi b(x-c)}
    \cos\left(\frac{\pi s}{\lambda L} (x-c)\right)
    \sin\left(\frac{\pi b}{\lambda L} (x-c)\right) \right)^2
\end{equation}
Beim Doppelspalt ergibt sich nach mehreren Versuchen mit verschiedenen
Startwerten kein geeigneter Fit der Kurve aus
Formel~\eqref{eq:double-slit-current}. Statt dessen werden die Werte, die
auf den verwendeten Spalten stehen, heran gezogen und in
Abbildung~\ref{fig:double-slit} die Meßwerte mit der theoretischen
Verteilung verglichen. Dazu setzen wir $s = \SI{0.33}{\milli\metre}$, $L
= \SI{1}{\metre}$, $\lambda = \SI{633}{\nano\metre}$ und $b =
\SI{0.08}{\milli\metre}$ in Formel~\eqref{eq:double-slit-current}
ein. Als Proportionalitätsfaktor wird \SI{0.2e-6}{\ampere\per\watt}
verwendet.

\begin{figure}
  \centering
  \includegraphics[width = 0.8\textwidth]{double-slit}
  \caption{Meßwerte und Graph der theoretischen Beugungsfigur beim Doppelspalt}
  \label{fig:double-slit}
\end{figure}
