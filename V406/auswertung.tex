% This work is licensed under the Creative Commons
% Attribution-NonCommercial 3.0 Unported License. To view a copy of this
% license, visit http://creativecommons.org/licenses/by-nc/3.0/.

\section{Auswertung}

\sisetup{ exponent-to-prefix = true }

Aus geometrischen Überlegungen zur Apparatur (siehe 
Abb.~\ref{fig:apparatur-geometrie}) kann ein Zusammenhang zwischen der
gemessenen Strecke $x$ und dem Beugungs-Winkel $\phi$ hergestellt
werden. Es gilt:
%
\begin{equation}
  \label{eq:x-phi}
  \sin\phi = \frac{x}{L}\text{.}
\end{equation}

\begin{figure}[b]
  \centering
  \includegraphics{geometrie}
  \caption{Geometrie der Apparatur}
  \label{fig:apparatur-geometrie}
\end{figure}

\subsection{Einzelspalt}

Bei der Messung am Einzelspalt, kommt ein Einzelspalt mit der Breite
\SI{0.08e-3}{\metre} (laut Aufdruck) zum Einsatz. Die Spaltbreite wird
zusätzlich mit einem Mikroskop gemessen, das im Objektiv eine Skala
enthält. Mithilfe eines Vergleichsobjektes wird die Breite eines
Skalenteils zu \SI{0.037e-3}{\metre} bestimmt. Die Breite des Spaltes
errechnet sich mit dieser Methode zu \SI{0.074e-3}{\metre}. 

Die gewonnenen Meßwerte (diese sind angehängt) werden nun in eine
nichtlineare Regression gegeben\footnote{hierzu wird die Funktion
  \texttt{scipy.optimize.curve\_fit} aus der \texttt{scipy}-Bibliothek
  in der Version 0.11 verwendet}. Aus
Formel~\eqref{eq:intensität_einzelspalt}, die die Intensität beschreibt,
und Formel~\eqref{eq:x-phi} ergibt sich dieser Zusammenhang zwischen der
gemessenen Stromstärke $I$ und der Strecke $x$
%
\begin{equation}
  \label{eq:strom-einzelspalt}
  I(x) \propto A_0\frac{L\lambda}{\pi(x-c)} \sin\left(\frac{\pi b}{\lambda
        L} (x-c)\right)
\end{equation}
%
Die nichtlineare Regression führt auf den Funktionsterm
%
\begin{equation}
  \label{eq:single-slit-fit}
  I(x) = \SI{8.828}{\ampere\per\metre}
  \frac{\lambda L}{\pi(x-\SI{27.2e-3}{\metre})}
  \sin\left(\frac{\pi\cdot\SI{0.0778e-3}{\metre}}{\lambda L} 
    (x-\SI{27.2e-3}{\metre})\right)
\end{equation}
%
mit den Werten $\lambda = \SI{633}{\nano\metre}$ und $L =
\SI{1}{\metre}$. Die Strecke $L$ ist durch Messung bekannt. Die
Wellenlänge des verwendeten He-Ne-Lasers ist in \textcite[36]{v406}
angegeben. In Abbildung~\ref{fig:single-slit} sind die Meßwerte und der
Graph der erhaltenen Funktion dargestellt.

\begin{figure}
  \centering
  \includegraphics[width = 0.8\textwidth]{single-slit}
  \caption{Meßwerte und Graph des Fits beim Einzelspalt}
  \label{fig:single-slit}
\end{figure}

\subsection{Variabler Spalt}

Für den variablen Spalt wird die gleiche Funktion aus
Formel~\eqref{eq:strom-einzelspalt} mit den erhaltenen Meßwerten
(ebenfalls im Anhang) in eine nichtlineare Regression gegeben. Es
wird Funktionsterm
%
\begin{equation}
  \label{eq:variable-slit-fit}
  I(x) = \SI{4.189}{\ampere\per\metre}
  \frac{\lambda L}{\pi(x-\SI{26.2e-3}{\metre})}
  \sin\left(\frac{\pi\cdot\SI{0.0503e-3}{\metre}}{\lambda L} 
    (x-\SI{26.2e-3}{\metre})\right)
\end{equation}
%
mit den Werten $\lambda = \SI{633}{\nano\metre}$ und $L =
\SI{1}{\metre}$ erhalten.  In Abbildung~\ref{fig:variable-slit} sind die
Meßwerte mit dem Graphen der erhaltenen Funktion dargestellt.

\begin{figure}
  \centering
  \includegraphics[width = 0.8\textwidth]{variable-slit}
  \caption{Meßwerte und Graph des Fits beim variablen Spalt}
  \label{fig:variable-slit}
\end{figure}

\subsection{Doppelspalt}

Auch wenn in Abbildung~\ref{fig:apparatur-geometrie} nur ein Einzelspalt
gezeichnet ist, ändert sich natürlich an der Beziehung zwischen $\phi$
und $x$ nichts, wenn man diesen durch einen Doppelspalt ersetzt. Aus
Formel~\eqref{eq:intensität_doppelspalt} und \eqref{eq:x-phi} ergibt
sich dann folgender Zusammenhang zwischen der gemessenen Strecke $x$ und
der Stromstärke $I(x)$:
%
\begin{equation}
  \label{eq:double-slit-current}
  I(x) \propto \left( \frac{2\lambda L}{\pi b(x-c)}
    \cos\left(\frac{\pi s}{\lambda L} (x-c)\right)
    \sin\left(\frac{\pi b}{\lambda L} (x-c)\right) \right)^2
\end{equation}
Beim Doppelspalt ergibt sich nach mehreren Versuchen mit verschiedenen
Startwerten kein geeigneter Fit der Kurve aus
Formel~\eqref{eq:double-slit-current}. Statt dessen werden die Werte,
die auf den verwendeten Spalten stehen, herangezogen und in
Abbildung~\ref{fig:double-slit} die Meßwerte mit der theoretischen
Verteilung verglichen. Dazu setzen wir $s = \SI{0.33e-3}{\metre}$, $L =
\SI{1}{\metre}$, $\lambda = \SI{633e-3}{\metre}$, $c =
\SI{26e-3}{\metre}$ und $b = \SI{0.08e-3}{\metre}$ in
Formel~\eqref{eq:double-slit-current} ein. Als Proportionalitätsfaktor
wird \SI[exponent-to-prefix = false,
per-mode=symbol]{2e-7}{\ampere\per\watt} verwendet.

\begin{figure}
  \centering
  \includegraphics[width = 0.8\textwidth]{double-slit}
  \caption{Meßwerte und Graph der theoretischen Beugungsfigur beim Doppelspalt}
  \label{fig:double-slit}
\end{figure}
