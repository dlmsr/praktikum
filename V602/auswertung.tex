% This work is licensed under the Creative Commons
% Attribution-NonCommercial 3.0 Unported License. To view a copy of this
% license, visit http://creativecommons.org/licenses/by-nc/3.0/.

\section{Auswertung}

\subsection{Überprüfung der \name{Bragg}-Bedingung}

Aus Tabelle~\ref{tab:bragg-bed} kann entnehmen werden, daß das Maximum
der Zählrate unter dem Zählrohr-Winkel \SI{27.9}{\degree} erreicht
ist. Das liefert für den Glanzwinkel
%
\begin{equation}
  \theta = \SI{13.95}{\degree} .
\end{equation}
%
Der Winkel weicht also nur um \SI{0.05}{\degree} vom erwarteten Winkel
von \SI{14}{\degree} ab. Damit ist die \name{Bragg}-Bedingung bestätigt.


\begin{table}
  \centering
  \begin{tabular}{SS}
    \toprule
    {$\theta/\si{\degree}$} & {Zählrate} \\
    \midrule
%    \input{bragg-bedingung}
    \bottomrule
  \end{tabular}
  \caption{Die Meßwerte zur Überprüfung der \name{Bragg}-Bedingung. Der
    angegebene Winkel ist der Zählrohrwinkel. Der Kristallwinkel war
    fest auf \SI{14}{\degree} eingestellt.}
  \label{tab:bragg-bed}
\end{table}

\subsection{Emissionspektrum der Kupferanode}



\subsection{Absorptionsspektren verschiedener Materialien}

