% This work is licensed under the Creative Commons
% Attribution-NonCommercial 3.0 Unported License. To view a copy of this
% license, visit http://creativecommons.org/licenses/by-nc/3.0/.

\section{Diskussion}

Die \name{Bragg}-Bedingung ist von allen Meßergebnissen am besten
nachgewiesen werden können. Eine absolute Abweichung von
\SI{0.05}{\degree} vom erwarteten Wert konnte erzielt werden.

Die Bestimmung der Abschirmzahlen gelingt gut. In
Tabelle~\ref{tab:abschirm-vergleich} können die ermittelten
Abschirmzahlen mit den vorher aus den
Literaturwerten\footcitetext[entnommen aus][]{esrf-abschirm} berechneten
Abschirmzahlen verglichen werden. Wie zu erkennen ist, weichen die
Abschirmzahlen bis auf Germanium, dessen Abschirmzahl mit einem relativ
geringen Fehler von nur \SI{7}{\percent} ermittelt worden ist, stark von
den aus den Literaturwerten berechneten ab. Während der Messungen ist
schon ein systematischer Fehler von \SIrange{1}{2}{\degree}
aufgefallen. Möglicherweise liegt es daran.

\begin{table}
  \centering
  \begin{tabular}{lSSSS}
    \toprule
    & {berechn. $\sigma_\mathrm{K}$} &
    {ermittelt. $\sigma_\mathrm{K}$} &
    {abs. Fehler} & {rel. Fehler in \si{\percent}}\\
    \midrule
    Ge (Germanium)     & 3.68 & 3.94 & 0.26 & 7    \\
    Nb (Niob)          & 4.13 & 5.43 & 1.30 & 31.5 \\
    Rb (Rubidium)      & 3.94 & 4.79 & 0.85 & 21.6 \\
    Zn (Zink)          & 3.56 & 4.04 & 0.48 & 13.5 \\
    Zr (Zirkonium)     & 4.09 & 5.25 & 1.16 & 28.4 \\
    \bottomrule
  \end{tabular}
  \caption{Vergleich der ermittelten Abschirmzahlen und den 
    berechneten Abschirmzahlen.}
  \label{tab:abschirm-vergleich}
\end{table}

Die relative Abweichung der Abschirmzahl von Gold (erwarteter Wert
$\sigma_\mathrm{L} = \num{3.5}$) beträgt \SI{42.6}{\percent}. Dieser
Fehler ist wohlmöglich durch den obengenannten systematischen Fehler
(ca. \SI{0.8}{\degree}) der ermittelten Winkel zu erklären. Subtrahiert
man diesen Fehler von den beiden Winkeln und errechnet dann die
Abschirmzahl erhält man $\sigma_\mathrm{L} = \num{3.07}$, womit sich
eine relative Abweichung von nur \SI{12.3}{\percent} ergibt. Ähnliche
Verbesserungen würden vermutlich auf bei den Abschirmzahlen der anderen
Absorber beobachtet werden, subtrahierte man die \SI{0.8}{\degree}.

Der Literaturwert für die Rydberg-Energie ist aus
\textcite{codata:rydhcev} entnommen und lautet
%
\begin{equation}
  h c R_\infty = \SI{13.6}{\electronvolt}.
\end{equation}
%
Damit beträgt die relative Abweichung des gemessenen vom erwarteten Wert
\SI{20.5}{\percent}. Auch hier läßt sich die Abweichung mit dem
systematischen Fehler erklären, mit denen die Winkel der anderen
Absorber behaftet sind.

