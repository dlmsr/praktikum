% This work is licensed under the Creative Commons
% Attribution-NonCommercial 3.0 Unported License. To view a copy of this
% license, visit http://creativecommons.org/licenses/by-nc/3.0/.

\section{Auswertung}

Aus den gewonnenen Daten, die nachfolgend vorgestellt werden, kann man
den Elastizitätsmodul der untersuchten Metalle und Legierungen
errechnen. Um eine lineare Regression\footnote{benutzt wurde hierfür die
  \texttt{ipython}-Umgebung mit der \texttt{numpy}-Bibliothek in der Version
  1.6.2} anwenden zu können, stellt man die Zusammenhänge
\eqref{eq:durchbiegung-einseitig} und \eqref{eq:durchbiegung-beidseitig}
so dar:
%
\begin{align}
  \label{eq:lineare-durchbiegung-einseitig}
  D(x) &= f(Lx^2-\frac{x^3}{3}) & f(u) &:= \frac{F}{2EI}u
\end{align}
Dies ist für die einseitige Aufhängung des Stabes. Für die beidseitige
Aufhängung ergibt sich:
\begin{align}
  \label{eq:lineare-durchbiegung-beidseitig}
  D(x) &= \begin{cases}
    g(3L^2x-4x^3) & 0\le x\le \frac{L}{2}\\
    g(4x^3-12Lx^2 + 9L^2x - L^3) & \frac{L}{2}\le x\le L
  \end{cases}
  & g(u) &:= \frac{F}{48EI}u
\end{align}
%
Hier können nun die Größen $A_g = F/(48EI)$ und $A_f = F/(2EI)$ bestimmt
werden. Umstellen von der beiden Größen nach $E$ liefert:
\begin{align}
  E &= \frac{F}{2A_fI} & E =& \frac{F}{48A_gI}
\end{align}
Da diese fehlerbehaftet sind, wird eine \name{Gauß}-Fehlerfortpflanzung
durchgeführt:
%
\begin{equation}
  \label{eq:gaussfehler-f}
  \Delta E = \sqrt{\left(\frac{\partial E}{\partial A_f}\cdot\Delta A_f\right)^2}
  = \frac{F\cdot\Delta A_f}{2A_f^2I}
\end{equation}
%
\begin{equation}
  \label{eq:gaussfehler-g}
  \Delta E = \sqrt{\left(\frac{\partial E}{\partial A_g}\cdot\Delta
      A_g\right)^2}
  = \frac{F\cdot\Delta A_g}{48A^2_gI}
\end{equation}

\subsection{Bestimmung des Elastizitätsmoduls von Stahl}

Die gemessenen Durchbiegung in Abhängigkeit vom Ort können in der
Tabelle~\ref{tab:stahl} nachgesehen werden. Es wurden immer die
Durchbiegungen $D_v(x)$ ohne Gewicht und die Durchbiegungen $D_n(x)$ mit
Gewicht bestimmt. Gemäß dieser Formel errechnet sich die effektive
Durchbiegung~$D(x)$, die Grundlage für weitere Berechnungen ist:
%
\begin{equation}
  D(x) = D_n(x) - D_v(x)
\end{equation}
%
Die Werte zur Bestimmung des Durchmessers $d$ findet man in
Tabelle~\ref{tab:stahl-durchmesser}. Er wurde zu folgendem Wert
bestimmt:
%
\begin{equation}
  d = \SI{9.985(3)}{\milli\meter}
\end{equation}
%
Es ergibt sich nach Regression und Fehlerrechnung folgender Wert für
$E$:
%
\begin{equation}
  \label{eq:wert-stahl-emodul}
  E = \SI{1077(53)}{\kilo\newton\per\milli\metre}
\end{equation}
%

\begin{table}
  \centering\small
  \begin{tabular}{SSS}
    \toprule
    {$x/\si{\centi\metre}$} &
    {$D_v/\si{\micro\metre}$} &
    {$D_n/\si{\micro\metre}$} \\
    \midrule
    2.7 &    550 &    510 \\
    5.0 &    600 &    470 \\
    9.0 &    600 &    220 \\
    13.0 &    590 &    185 \\
    17.0 &    540 &    136 \\
    21.0 &    530 &    280 \\
    25.0 &    460 &    312 \\
    27.0 &    430 &    377 \\
    32.0 &    310 &    475 \\
    35.0 &    200 &    490 \\
    38.0 &    170 &    544 \\
    40.0 &    198 &    600 \\
    42.0 &    198 &    647 \\
    44.0 &    195 &    700 \\
    46.0 &    180 &    749 \\
    48.0 &    177 &    795 \\
    \bottomrule
  \end{tabular}
  \caption{Meßwerte zur Durchbiegung des Stahlstabes}
  \label{tab:stahl}
\end{table}

\begin{table}
  \centering\small
  \begin{tabular}{S}
    \toprule
    {$d/\si{\milli\metre}$}\\
    \midrule
    9.986 \\
    9.989 \\
    9.983 \\
    9.988 \\
    9.986 \\
    9.982 \\
    9.982 \\
    9.985 \\
    9.985 \\
    9.981 \\
    \bottomrule
  \end{tabular}
  \caption{Durchmesser des Stahlstabes}
  \label{tab:stahl-durchmesser}
\end{table}

\subsection{Bestimmung des Elastizitätsmoduls von Aluminium}

Die gemessenen Durchbiegungen für den Aluminiumstab ist in
Tabelle~\ref{tab:aluminium} zu finden. Die Werte zum Durchmesser in
Tabelle~\ref{tab:aluminium-durchmesser}. Für diesen ergibt sich:
%
\begin{equation}
  d = \SI{9.985(3)}{\milli\metre}
\end{equation}
%
Nach der Regression und der Fehlerrechnung läßt sich ein Wert für den
E-Modul angeben:
%
\begin{equation}
  E = \SI{73.0(7)}{\kilo\newton\per\milli\metre}
\end{equation}

\begin{table}
  \centering\small
  \begin{tabular}{SSS}
    \toprule
    {$x/\si{\centi\metre}$} & 
    {$D_v/\si{\micro\metre}$} & 
    {$D_n/\si{\micro\metre} $} \\
    \midrule
    2.7  &   920   &    930 \\
    7.5  &  1030   &  1170  \\
    12.5 &  1210   &  1550  \\
    17.5 &  1430   &  2060  \\
    22.5 &  1680   &  2680  \\
    27.5 &  2020   &  3410  \\
    30.0 &  2220   &  3800  \\
    32.0 &  2350   &  4150  \\
    34.0 &  2520   &  4480  \\
    36.0 &  2710   &  4850  \\
    38.0 &  2850   &  5230  \\
    40.0 &  3030   &  5610  \\
    42.0 &  3190   &  6000  \\
    44.0 &  3320   &  6410  \\
    46.0 &  3660   &  6770  \\
    48.0 &  3800   &  7240  \\
    \bottomrule
  \end{tabular}
  \caption{Meßwerte zur Durchbiegung des Aluminiumstabes}
  \label{tab:aluminium}
\end{table}

\begin{table}
  \centering\small
  \begin{tabular}{S}
    \toprule
    {$d/\si{\milli\metre}$}\\
    \midrule
    9.984 \\
    9.988 \\
    9.984 \\
    9.971 \\
    9.979 \\
    9.980 \\
    9.994 \\
    9.986 \\
    9.988 \\
    9.982 \\
    \bottomrule
  \end{tabular}
  \caption{Durchmesserx des Aluminiumstabes}
  \label{tab:aluminium-durchmesser}
\end{table}

\subsection{Bestimmung des Elastizitätsmoduls von Messing}



%%% Local Variables: 
%%% mode: latex
%%% TeX-master: "protokoll"
%%% End: 
