% This work is licensed under the Creative Commons
% Attribution-NonCommercial 3.0 Unported License. To view a copy of this
% license, visit http://creativecommons.org/licenses/by-nc/3.0/.

\section{Theorie}

Kräfte können Körper nicht nur beschleunigen, sondern auch
verformen. Zumeist betrachtet man diese Kräfte dann auf die Fläche, an
der sie angreifen, normiert und nennt die so erhaltene Größe
\emph{Spannung}. Die zur Oberfläche senkrechte Komponente heißt
\emph{Normalspannung}, die parallele \emph{Tangentialspannung}.

\subsection{Der Elastizitätsmodul}

Wenn die relative Längenänderung $\Delta L/L$ eines Körpers unter
Einfluß obengenannter Spannungen hinreichend klein ist, so kann man
einen linearen Zusammenhang, der häufig auch als \name{Hook}sches Gesetz
bezeichnet wird, beobachten:
%
\begin{equation}
  \label{eq:hook's-law}
  \sigma = E\frac{\Delta L}{L}
\end{equation}
%
Hier bezeichnet $\sigma$ die Normalspannung, die in Richtung der
Längenausdehnung $L$ des Körpers wirkt. Mit $E$ bezeichnet man den
sogenannten \emph{Elastizitätsmodul}. Dieser ist vom Material abhängig.

\subsection{Durchbiegung eines homogenen Stabes bei einseitiger
  Einspannung}

Wird ein Stab mit Masse $m$ und Länge $L$ an seinem einen Ende fest
eingespannt und an seinem anderen Ende mit einem Gewicht versehen, so
biegt er sich durch und es kann unter Betrachtung der auftretenden
Drehmomente folgende Beziehung für die Durchbiegung $D(x)$ hergeleitet
werden:
%
\begin{equation}
  \label{eq:durchbiegung-einseitig}
  D(x) = \frac{mg}{2EI}\left(Lx^2 - \frac{x^3}{3}\right), \quad (0\le
  x\le L)
\end{equation}
%
Der Buchstabe $E$ bezeichnet hier den zuvor erwähnten Elastizitätsmodul
und $I$ bezeichnet das Flächenträgheitsmoment, das sich wie folgt
bestimmt:
%
\begin{equation}
  \label{eq:flaechentraeg.moment}
  I = \int_Q y^2\:\mathrm{d}(x, y)
\end{equation}
%
Hier bezeichnet $Q$ die Querschnittsfläche des Stabes und $y$ den
Abstand zur sogenannten \emph{neutralen Faser}. Dies ist eine gedachte
Schnittlinie mit derjenigen Ebene, in der keine Spannungen auftreten,
in der also die Länge des Stabes nicht verändert wird.

\subsection{Durchbiegung eines homogenen Stabes bei beidseitiger
  Einspannung}

Betrachtet man nun denselben Stab an beiden Enden eingespannt und in
seiner Mitte durch ein Gewicht belastet, so kommt man unter ähnlicher
Betrachtung der auftretenden Drehmomente wieder zu einem Zusammenhang
zwischen der Durchbiegung $D(x)$ und dem Ort $x$:
%
\begin{equation}
  \label{eq:durchbiegung-beidseitig}
  D(x) = \frac{F}{48 E I} \cdot 
  \begin{cases}
    (3 L^2 x - 4 x^3) & 0\le x\le \frac{L}{2}\\
    (4 x^3 - 12 L x^2 + 9 L^2 x - L^3) & \frac{L}{2}\le x\le L
  \end{cases}
\end{equation}

%%% Local Variables: 
%%% mode: latex
%%% TeX-master: "protokoll"
%%% End: 
