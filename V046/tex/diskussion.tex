% This work is licensed under the Creative Commons
% Attribution-NonCommercial 3.0 Unported License. To view a copy of this
% license, visit http://creativecommons.org/licenses/by-nc/3.0/.

\section{Diskussion}
Die Magnetfeldmessung zeigt die erwarteten Ergebnisse. Das Feld ist in
der Nähe der Probe maximal und zeigt über einen sehr kleinen Bereich
Homogenität.  Weit außerhalb fällt das Feld stark ab.

Die Bestimmung der Faraday-Rotation ist nicht gut gelungen.  Die
Meßfehler sind ziemlich groß und wie man in \cref{fig:linregress}
erkennen kann, streuen die Punkte stark um die Regressionsgerade. Das
Verhältnis $m^*/m_\text{e}$ ist in \cref{ecee-colorado} für
Galliumarsenid mit \num{0.067} angegeben.  Unser Wert weicht mit
\num{0.290} um \SI{333}{\percent} ab.  Das läßt sich zum Teil auf
die beschädigten Interferenzfilter zurückführen, die zum Teil Kratzer
oder Löcher aufwiesen.  Eine schlechte Justage der Apparatur könnte
ebenfalls eine Rolle gespielt haben.
