% This work is licensed under the Creative Commons
% Attribution-NonCommercial 3.0 Unported License. To view a copy of this
% license, visit http://creativecommons.org/licenses/by-nc/3.0/.

\section{Theorie}
\subsection{Zur effektiven Masse}
%
Freie Elektronen der Masse $m_\text{e}$ gehorchen der 
Dispersionsrelation in Gleichung~\eqref{eq:dispersion_frei}. 
Das $\epsilon$ bezeichnet dabei die Energie des Elektrons, $k$ die 
De-Broglie-Wellenzahl und $\hbar$ das reduzierte Planck'sche 
Wirkungsquantum.

\begin{equation}
\epsilon = \frac{\hbar^2k^2}{2m_\text{e}}
\label{eq:dispersion_frei}
\end{equation}

Befindet sich ein Elektron hingegen in einem Kristall, so 
lässt sich dies näherungsweise dadurch beschreiben, dass 
der Raumbereich, in welchem sich das Elektron bewegen darf 
auf das Kristallvolumen beschränkt ist, und dass sich im 
Kristallinneren ein periodisches Potential befindet. 

Ein solches Elektron kann ähnlich wie ein freies Elektron beschrieben 
werden, wenn anstelle der Elektronenmasse $m_\text{e}$ eine 
sogenannte effektive Masse $m^*$ eingeführt wird. Dadurch bleibt 
die Schrödingergleichung identisch zu derjenigen freier Elektronen. 

Im Allgemeinen hängt die Energie eines Elektrons im Kristall auch von 
der Richtung des Wellenvektors ab, welcher im Kristall einen 
Einteilchenzustand bis auf den Spin charaktierisiert. Außerdem 
tritt Entartung auf, d.h. für einen Wellenvektor sind verschiedene 
Energien erlaubt. Es gibt Energiebereiche, welche für keinen 
Wellenvektor eingenommen werden. Diese Bereiche werden als 
Bandlücken bezeichnet. 

Bei Kenntnis der Bandstruktur kann die effektive Masse eines 
Kristallelektrons durch Formel~\eqref{eq:effektivemasse} 
berechnet werden und ist im Allgemeinen richtungsabhängig.

\begin{equation}
m_i^* = \frac{\hbar^2}{\left(\pd{^2\epsilon}{k_i^2}\right)}
\label{eq:effektivemasse}
\end{equation}

Dies führt für nicht zu hohe elektrische und magnetische Felder zu 
dem Newtonschen Bewegungsgesetz der Form~\eqref{eq:newton}.
Also ist in diesem Fall eine klassiche Betrachtung für den 
Faraday-Effekt gerechtfertigt, solange die Elektronenmasse durch die 
effektive Elektronenmasse ersetzt wird.

\begin{equation}
F_i = m_i^*\cdot a_i
\label{eq:newton}
\end{equation}

%
\subsection{Zum Faraday-Effekt}
%

Der Faraday-Effekt bezeichnet die Drehung der Polarisationsachse von 
linear polarisiertem Licht in Materie, während ein äußeres Magnetfeld 
$\vec{B}$ in Richtung der Lichtausbreitung $\vec{e}_\text{z}$ 
angelegt ist.

Dieser Effekt kann erklärt werden, wenn das linear polarisierte Licht 
als Überlagerung einer links- und einer rechtszirkularen 
elektromagnetischen Welle beschrieben wird und es zwei 
verschiedene Brechungsindices für die beiden Wellenanteile in 
dem Kristall gibt. Kristalle, die diese Eigenschaft besitzen, werden 
als zirkular doppelbrechend bezeichnet.

Nach Formel~\eqref{eq:newton} kann die Bewegungsgleichung für 
ein gebundenes Elektron der Ladung $e_0$ und Masse $m$ durch die 
Differentialgleichung~\eqref{eq:bewgl} modelliert werden. 
Hierbei ist $\vec{r}$ die Auslenkung aus der Gleichgewichtslage, 
K die Rückstellkonstante durch die Bindung des Elektrons, 
$\vec{B}$ das von außen angelegte Magnetfeld und $\vec{E}$ die 
elektrische Feldstärke der elektromagnetischen Welle.

Im Folgenden seien die Ausrichtung des Magnetfeldes 
und die Ausbreitungsrichtung des Lichtes die 
z-Richtung.

\begin{equation}
m\frac{\d^2\vec{r}}{\d t^2} = -K\vec{r} - e_0\vec{E} -e_0
\frac{\d\vec{r}}{\d t} \times \vec{B}
\label{eq:bewgl}
\end{equation}

Ein harmonischer Ansatz der Feldstärke, wie in~\eqref{eq:harm} 
angegeben, liefert nach langer Rechnung 
einen Ausdruck für die dielektrische Suszeptibilität $\chi$. Diese 
beschreibt für nicht zu große elektrische Felder den Zusammenhang 
zwischen makroskopischer Polarisation $\vec{P}$ der Materie und 
angelegtem elektrischen Feld. Gleichung~\eqref{eq:polarisation} 
drückt dies noch einmal mathematisch aus, wobei $\chi$ im 
Allgemeinen ein Tensor zweiter Stufe ist.

\begin{equation}
\vec{E}(\vec{r},t) = \vec{E}_0\exp{(i(\vec{k}\vec{r} -\omega t))}
\label{eq:harm}
\end{equation}


\begin{equation}
\vec{P} = \epsilon_0\chi\vec{E}
=\epsilon_0
\left(
\begin{matrix}
\chi_{xx} & \chi_{xy} &\chi_{xz} \\
\chi_{yx} & \chi_{yy} & \chi_{yz} \\
\chi_{zx}&\chi_{zy} & \chi_{zz} \\
\end{matrix}
\right)
\cdot\vec{E}
\label{eq:polarisation}
\end{equation}

Es stellt sich heraus, dass der Eintrag $\chi_{xy}$ für den 
Faraday-Effekt besonders wichtig ist.
Die Rechnung ergibt für diesen den Ausdruck in~\eqref{eq:suszept}.
Dieser wird weiter unten benötigt.
Dabei bezeichnet $N$ die Elektronendichte des Materials.

\begin{equation}
\chi_{xy} = i\frac{Ne_0^3\omega B}{\epsilon_0\left((-m
\omega^2 + K)^2 - (e_0\omega B)^2\right)}
= -\chi_{yx}
\label{eq:suszept}
\end{equation}
%

Die Wellengleichung des Lichtes in Materie ist in~\eqref{eq:maxwell} 
angegeben. Der harmonische Ansatz~\eqref{eq:harm} ergibt zwei 
Wellen mit unterschiedlichen Wellenvektoren $k_+$ und $k_-$, von 
denen eine Welle links- und die andere rechtszirkular polarisiert 
ist, falls $\chi_{xy} = - \chi_{yx}$ gilt. Die Werte der Wellenzahlen 
sind in~\eqref{eq:kplusminus} angegeben.

\begin{equation}
\nabla \times (\nabla \times \vec{E}) = 
-\frac{1+\chi}{c^2}\pd{^2\vec{E}}{t^2}
\label{eq:maxwell}
\end{equation}

\begin{equation}
K_\pm = \frac{\omega}{c}\sqrt{(1+\chi_{xx}) \pm \frac{\chi_{xy}}{i}}
\label{eq:kplusminus}
\end{equation}

Die Überlagerung zweier entgegengesetzt zirkular laufender 
Wellen mit unterschiedlichen Wellenzahlen ist in 
Gleichung~\eqref{eq:super} zu sehen. 

\begin{equation}
\vec{E}(z) = \frac{1}{2}(\vec{E}_R(z) +\vec{E}_L(z))=
\frac{1}{2}\left((E_0\vec{e}_x - iE_0\vec{e}_y)e^{ik_R\cdot z}+
(E_0\vec{e}_x +iE_0\vec{e}_y)e^{ik_L\cdot z}\right)
\label{eq:super}
\end{equation}

Umschreiben dieser 
Gleichung liefert den in Formel~\eqref{eq:winkela} angegebenen 
Ausdruck für den Winkel $\Theta$, um welchen sich die 
Polarisationsachse des einfallenden Lichtes nach durchlaufen 
des Mediums der Länge L gedreht hat.

\begin{equation}
\Theta = \frac{L}{2}(k_R - k_L)
\label{eq:winkela}
\end{equation}

Werden hierfür die in den Gleichungen~\eqref{eq:kplusminus} 
angegebenen Wellenzahlen eingesetzt, der Ausdruck für einen in diesem 
Versuch erlaubten Fall genähert, sowie die Suszeptibilität 
aus~\eqref{eq:suszept} eingesetzt, folgt schlussendlich die 
Formel~\eqref{eq:winkel} für den Winkel $\Theta$. 

\begin{equation}
\Theta = \frac{e_0^3}{2\epsilon_0c}\frac{\omega^2}{(-m\omega^2 
+ K)^2 - (e_0\omega B)^2}
\frac{NBL}{n}
\label{eq:winkel}
\end{equation}

Mit n wird der Brechnungsindex des verwendeten Materials ohne 
angelegtes Magnetfeld bezeichnet. Der Ausdruck $\sqrt{1+\chi_{xx}}$ 
wurde durch einen Ausdruck ersetzt, der den Brechungsindex besitzt.

Eine Näherung für die in diesem Versuch verwendeten Messfrequenzen, 
sowie der Übergang
\begin{equation*}
K \rightarrow 0 
\end{equation*}
liefern den Einfluss der quasifreien Bandelektronen auf den 
Drehwinkel. Dieser Einfluss wird mit $\Theta_\text{frei}$ 
bezeichnet und ist in Formel~\eqref{eq:winkelfrei} angegeben.

\begin{equation}
\Theta_\text{frei} = \frac{e_0^3\lambda^2}{8\pi^2\epsilon_0c^3m^2}
\frac{NBL}{n}
\label{eq:winkelfrei}
\end{equation}

In dieser Formel ist $\lambda$ die Wellenlänge des einfallenden 
Lichtes.