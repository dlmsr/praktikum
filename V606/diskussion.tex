% This work is licensed under the Creative Commons
% Attribution-NonCommercial 3.0 Unported License. To view a copy of this
% license, visit http://creativecommons.org/licenses/by-nc/3.0/.

\section{Diskussion}
%
Zum Schluss sollen die Ergebnisse dieses Versuches diskutiert werden.
Die bestimmte Güte des Selektivverstärkers beträgt $Q=\num{118}$.  Am
Selektivverstärker wurde eine Güte von 100 eingestellt.  Die relative
Abweichung der beiden Werte beträgt \SI{18}{\percent}.  Es wurden in
diesem Versuch nicht direkt die Ausgangsspannungen bei den Frequenzen
gemessen, bei denen die Spannung auf $\frac{1}{\sqrt{2}}$ der
Höchstspannung gefallen ist, sodass relativ grobe Abschätzungen gemacht
werden mussten.  Außerdem ist es nicht möglich einen Fehler für die
bestimmte Güte anzugeben.  Es wird also festgestellt, dass der
verwendete Selektivversätker tatsächlich die eingestellte Güte besitzt.

Bei der Bestimmung der Suszeptibilitäten der verwendeten Proben ergeben
sich nur geringe statistische Fehler.  Diese besitzen aufgrund der
geringen Messwerteanzahl nur eine schwache Aussagekraft und dürfen nicht
überschätzt werden. Die Werte für die Suszeptibilität der untersuchten
Proben, die sich bei den verschiedenen Verfahren ergeben, weichen
deutlich stärker voneinander ab als die errechneten statistischen Fehler
es angeben.
%
