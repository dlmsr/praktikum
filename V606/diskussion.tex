% This work is licensed under the Creative Commons
% Attribution-NonCommercial 3.0 Unported License.  To view a copy of
% this license, visit http://creativecommons.org/licenses/by-nc/3.0/.

\section{Diskussion}
%
Zum Schluss sollen die Ergebnisse dieses Versuches diskutiert werden.
Die bestimmte Güte des Selektivverstärkers beträgt $Q=\num{118}$.  Am
Selektivverstärker wurde eine Güte von 100 eingestellt.  Die relative
Abweichung der beiden Werte beträgt \SI{18}{\percent}.  Es wurden in
diesem Versuch nicht direkt die Ausgangsspannungen bei den Frequenzen
gemessen, bei denen die Spannung auf $\frac{1}{\sqrt{2}}$ der
Höchstspannung gefallen ist, sodass relativ grobe Abschätzungen gemacht
werden mussten.  Außerdem ist es nicht möglich einen Fehler für die
bestimmte Güte anzugeben.  Es wird also festgestellt, dass der
verwendete Selektivverstärker tatsächlich die eingestellte Güte besitzt.

Bei der Bestimmung der Suszeptibilitäten der verwendeten Proben ergeben
sich nur geringe statistische Fehler.  Diese besitzen aufgrund der
geringen Messwerteanzahl nur eine schwache Aussagekraft und dürfen nicht
überschätzt werden. Die Werte für die Suszeptibilität der untersuchten
Proben, die sich bei den verschiedenen Verfahren ergeben, weichen
deutlich stärker voneinander ab als die errechneten statistischen Fehler
es angeben.

Um die Vorhersagen der Quantenmechanik auf ihre Genauigkeit zu
überprüfen sollen hier noch einmal die theoretischen Werte mit den
beiden experimentell bestimmten verglichen werden.  Die Ergebnisse sind
in \cref{tab:vergleich} zu sehen.

\begin{table}
  \centering
  \begin{tabular}{lSSSSS}
    \toprule
    & {$\chi_U$} & {$\chi_R$} & {$\chi$} &
    {$\frac{\chi_U-\chi}{\chi}$ in \si{\percent}} &
    {$\frac{\chi_R-\chi}{\chi}$ in \si{\percent}} \\
    \midrule
    Dy$_2$O$_3$ & 0.0234 &  0.0186 & 0.0251 &  6.8 & 25.9\\
    Nd$_2$O$_3$ & 0.0054 &  0.0039 & 0.0030 & 80.0 & 30.0\\
    Gd$_2$O$_3$ & 0.0130 &  0.0108 & 0.0136 &  4.4 & 20.6\\
    \bottomrule
  \end{tabular}
  \caption{Die experimentell bestimmten Werte im Vergleich mit den
    theoretischen Vorhersagen.  }
  \label{tab:vergleich}
\end{table}

Insgesamt werden die Vorhersagen gut bestätigt.  Die Größenordnungen
stimmen bei allen Messungen mit den Vorhersagen überein.  Die relativen
Fehler der Messung mit der Spannungsmethode sind außer beim
Neodymium(III)-oxid sehr gering.  Bei der Widerstandsmethode sind die
Fehler zwar insgesamt höher, dafür ist aber auch beim Neodym die
Abweichung geringer.

\FloatBarrier