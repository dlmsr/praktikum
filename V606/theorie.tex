% This work is licensed under the Creative Commons
% Attribution-NonCommercial 3.0 Unported License. To view a copy of this
% license, visit http://creativecommons.org/licenses/by-nc/3.0/.

\section{Theorie}

\subsection{Magnetische Suszeptibilität und Paramagnetismus}

Zur Beschreibung des magnetischen Feldes im Vakuum wird die magnetische
Flußdichte~$\vec{B}$ verwendet. Sie erfüllt die
\name{Maxwell}-Gleichungen für das Vakuum. Werden nun Felder in Materie
studiert, ist es zweckmäßig eine neue Größe, die magnetische
Feldstärke~$\vec{H}$, so einzuführen:
\begin{equation}
  \vec{H} = \frac{1}{\mu_0}(\vec{B} - \vec{M}).
\end{equation}
Hierbei ist $\vec{M}$ die sogenannte Magnetisierung, die von der Materie
abhängig ist. Durch Einführung dieser Größe haben die
\name{Maxwell}-Gleichungen in Materie wieder dieselbe Form wie im
Vakuum. Im wesentlichen ist sie ein Mittelwert der im Material
vorkommenden magnetischen Momente~$\mu$. Es gilt:
\begin{equation}
  \vec{M} = \mu_0 \chi \vec{H}.
\end{equation}
Die Größe~$\chi$ heißt Suszeptibilität und ist keine Konstante, sondern
von Temperatur~$T$ und magnetischer Feldstärke~$\vec{H}$ abhängig.

Ab einer magnetischen Suszeptibilität~$\chi>0$ spricht man von
Paramagnetismus. Dieser tritt nur bei Materie auf, deren Bausteine (wie
z.\,B. Atome, Moleküle) einen Drehimpuls aufweisen. Damit ist eine
Ausrichtung der mit dem Drehimpuls verknüpften magnetischen Momente in
einem äußeren Magnetfeld verbunden. Diese Ausrichtung wird durch
thermische Bewegungen gestört, so daß der Paramagnetismus ein
temperaturabhängiger Effekt ist.

\subsection{Suszeptibilität der Verbindungen seltener Erden}
