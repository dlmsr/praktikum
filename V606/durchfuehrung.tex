% This work is licensed under the Creative Commons
% Attribution-NonCommercial 3.0 Unported License. To view a copy of this
% license, visit http://creativecommons.org/licenses/by-nc/3.0/.

\section{Aufbau und Durchführung}

\subsection{Aufbau einer Apparatur zur Bestimmung der Suszeptibilität}

Die Apparatur besteht im wesentlichen aus vier Teilen: Einem
Signalgeber, einer Brückenschaltung, einem Verstärker und einem
Meßgerät. Der Signalgeber speist die Brückenschaltung mit einem
Sinussignal, ein Verstärker leitet das Signal verstärkt an das Meßgerät
weiter. Ein Schema der Versuchsanordnung ist in \cref{fig:schema-aufbau}
skizziert. 

Das Prinzip, nach dem die Suszeptibilität bestimmt wird, basiert auf
einer Induktivitätsmessung einer Spule, in welche die zu messende Probe
eingeführt wird.

\subsection{Messung der Güte des Selektivverstärkers}

\subsection{Messung der Induktivität der Proben}

